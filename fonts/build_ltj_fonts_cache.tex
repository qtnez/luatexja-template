% Example to build luatex chinese fonts cache.
% UTF-8 for luatex.
\input luatexja-core.sty
\jfont\ming={I.MingCP:jfm=zh_TW/quanjiao} at 11pt
\jfont\hei={Taipei Sans TC Beta Light:jfm=zh_TW/quanjiao} at 11pt
\jfont\heib={Noto Sans CJK TC Medium:jfm=zh_TW/quanjiao} at 11pt
\jfont\kai={TW-MOE-Std-Kai:jfm=zh_TW/quanjiao} at 11pt
\jfont\iyan={I.Ngaan:jfm=zh_TW/quanjiao} at 11pt
\jfont\fangsong={cwTeXFangSong:jfm=zh_TW/quanjiao} at 11pt
\jfont\hminga={HanaMinA:jfm=zh_TW/quanjiao} at 11pt
\jfont\hmingb={HanaMinB:jfm=zh_TW/quanjiao} at 11pt

\input opmac
\margins/1 a4 (1,1,1,1)in  % A4 portrait
\fontfam[Linux Libertine]
\typosize[11/16.4]

\def\mytext{
「老兵不死,只是逐漸凋零」。用plain \TeX{}的最大好處是,二十年前的文件,
現在仍然可以正常編譯,幾乎不必修改。沒錯!\TeX{}就是這麼的「頑固」。
這是\LaTeX{}文件所不能及的。\TeX{}是老兵,但卻是「老驥伏櫪,志在千里」,
絕非「逐漸凋零」。}

\ming
\mytext

\hei
\mytext

\heib
\mytext

\kai
\mytext

\iyan
\mytext

\fangsong
\mytext

\hminga
犇軲晗堃煊

\hmingb
𤧞𠃝𠅀𠅘𫝁𫜁

\bye

