% !TEX program = LuaLaTeX + luatex-ja
% Copyright © 2021 Guoo Jehng Lee <edt1023@gmail.com>
% CC BY 4.0 license.
% Chat: https://t.me/TWTUG  (Taiwan TeX User Group)

\documentclass[11pt,a4paper]{article}
% which package used.
%\listfiles
\usepackage[a4paper,textwidth=410pt,textheight=660pt]{geometry}
% 定義 input 路徑
\makeatletter
  \def\input@path{{./pieces/}}
% \def\input@path{{./figures/}{./pieces/}}
\makeatother

% siunitx 和 physics 會有衝突 \qty 要用 \SI 取代。
\usepackage[svgnames]{xcolor}
\usepackage{my-zhfonts,parskip,mdframed,mparhack,amsmath,siunitx,textcomp,setspace,
  upgreek}
\usepackage[time=Chinese,style={Traditional,Normal}]{zhnumber}
\usepackage[nomath]{libertinus}
%\usepackage[math-style=ISO]{unicode-math}

% 定義正誤條邊框顏色及「回目錄」 button 顏色
\definecolor{britishracinggreen}{rgb}{0.0, 0.26, 0.15}
\definecolor{bulgarianrose}{rgb}{0.28, 0.02, 0.03}

% 正誤條的環境定義:
% 章節後不會留有正常空白,需要 startcode 校正。
\newmdenv[linecolor=red,frametitle=誤:,leftmargin=1em,rightmargin=1em,
  startcode=\leavevmode,topline=false,bottomline=false,linewidth=1.2pt,
  skipabove=2ex]{Wrong}
\newmdenv[linecolor=britishracinggreen,frametitle=正:,leftmargin=1em,
  rightmargin=1em,topline=false,bottomline=false,linewidth=1.2pt,
  skipabove=2ex,skipbelow=1ex]{Right}

% 定義程式碼
\newmdenv[leftmargin=1em,rightmargin=1em,linecolor=LightGrey!60,
  linewidth=1.5pt,skipabove=2ex,skipbelow=.5ex,
  backgroundcolor=LightGrey!20]{Code}

% 回目錄的定義
\newcommand{\back}{%
  \begin{flushright}
  \hyperlink{contents}{\textit{\textbf{\textcolor{bulgarianrose}{回目錄}}}}
  \end{flushright}}

\usepackage[pdfstartview={FitH},pdfencoding={auto},
  pdfauthor={李果正 Guoo Jehng Lee},pdfsubject={LaTeX 正誤手冊},
  pdftitle={LaTeX 正誤手冊},pdfkeywords={LaTeX, typesetting, conventions},
  bookmarksopen={true}]{hyperref}

% \zhtoday 定義於 ltj-zhfonts.sty。
%\title{\textsf{\textbf{\LaTeX 正誤手冊}}}
\title{\textsf{\textbf{\LaTeX \iyan 正誤手冊}}}
\author{\small 李果正 Guoo Jehng Lee}
\date{\small \zhtoday}

%%%%%%%%%%%%%%%%%%%%%%%%%%%%
%%%%%%%%% 本文開始 %%%%%%%%%
%%%%%%%%%%%%%%%%%%%%%%%%%%%%

\begin{document}

\maketitle

排版有許多慣例,甚至最後形成標準。這份文件主要就是排除違反慣例的情況。當
然所謂慣例,除非形成標準,要不然仍然是會有爭議(就算是形成了標準,也還是
有些人不願意遵守),這無所謂對、錯,這份文件的「正誤」也只是針對多數人遵
從的慣例而言,不是完全的非黑即白。

但是關於\TeX/\LaTeX 的語法,這就是非黑即白了,語法錯誤,嚴重的會使編譯中
止,文件出不來,輕一點的是排版結果不符合預期。因此這份文件的所謂正誤,也
包括了語法上的錯誤。

\section{基礎語法}
\label{sec:syntax}

%這一節是關於一些較基本、普遍的排版慣例、規範。也包括\LaTeX 的基礎語法的
%錯用。以下每個小節都是一個獨立的單元,為了索引方便,在每一小節末的右手邊
%,會有一個「回目錄」的文字 button,可以馬上回到目錄,利於查找,至於到了
%目錄,目錄本身的各小節本就有超連結的功能。為了閱讀方便,目錄是放在文件的
%最後,當成索引使用。

  \subsection{單位前需要小空白}
\label{sub:unit}

\begin{Wrong}
\begin{verbatim}
台灣南北的長度大約 394km,東西上大闊度大約 144km。
\end{verbatim}

台灣南北的長度大約 394km,東西上大闊度大約 144km。
\end{Wrong}

\begin{Right}
\begin{verbatim}
台灣南北的長度大約 394\,km,東西上大闊度大約 144\,km。
\end{verbatim}

台灣南北的長度大約 394\,km,東西上大闊度大約 144\,km。
\end{Right}

單位前需要一個小空白(插入\verb|\,|)。可以參考\href{https://github.com/josephwright/siunitx}{\sf siunitx}套件的例子。
如果不想傷這個腦筋,可以引用{\sf siunitx}套件,依照它的使用方法來表現數字及單位,
這樣就可以全文一致,不必一個一個去手動修正。

有規則就會有例外,非字母的符號(準)單位,例如溫度(24\textcelsius)、百分號(10\%),數字和符號間是不留空白的。a.m./p.m.\/這類表示上下午的也不留空白\footnote{正式文件一般主張要留空白,2:45\,p.m.,而且時區要用小括號括住,小括號前要留空白,2:45\,p.m.\,(EST)。}。

{\sf SI}(Système International d'Unités)對此有異議,認為只要是單位就得留白(\SI{24}{\degreeCelsius}、\SI{10}{\percent}),例外是單純的角度符號(\ang{30;8;22})就需要緊密。另外 \%\ 並非是{\sf SI}中認定的單位(但{\sf siunitx}中有定義百分號)。
在一般的寫作文件,對溫度度數及百分號不少人還是持不要空白的風格\footnote{Chicago style主張不要空白,APA及ACS style主張要空白。請參考:\url{https://blogs.millersville.edu/bduncan/numbers/}。}。

好像有點亂,所以寫正式文件時,有style manual的話,還是要詳細讀一讀,upstream的要求比你的喜好重要。如果不知何去何從,最簡單的方式就是依標準來。


  \marginpar{\back}

  \subsection{一些特殊字元不能直接鍵入}
\label{sub:special}

\begin{Wrong}
\begin{verbatim}
TeX 裡頭有一些特殊字元是無法直接鍵入的,例如倒斜線 \,那是指令的引頭字元,直接鍵入編譯也不會過。
\end{verbatim}
\end{Wrong}

\begin{Right}
\begin{verbatim}
那麼這類字元要如何鍵入呢?可以使用 \ 去 escape 它,但唯獨這個倒斜線不行,要用 \textbackslash 來鍵入,或者進入數學模式 $\backslash$,這些指令都滿長的。另有簡單的方式,就是直接取出字元來 \char`\\ 就可以了(那個 ` 是左單引號)。
\end{verbatim}
\end{Right}

以下說明各種特殊符號的鍵入方式。至於取巧用 \verb|\char`\\| 的方式,在章節標題時
最好不用,因為 pdf bookmarks 的顯示會不正常(無法正確轉換為ascii或UTF-16BE)。

\vspace{.5\baselineskip}
\begin{center}
\setstretch{1.2}
\begin{tabular}{llll}
\hline
符號 & 作用 & 文稿上使用 & \LaTeX\ 的替代指令 \\
\hline
\textbackslash & 下排版命令 & \verb|$\backslash$| & \verb|\textbackslash|\\
\%             & 註解       & \verb|\%|           & NA \\
\#             & 定義巨集   & \verb|\#|           & NA \\
\~{}           & 產生一個空白   & \verb|\~{}|     & \verb|\textasciitilde| \\
\$             & 進入(離開)數學模式 & \verb|\$| & \verb|\textdollar| \\
\_{}           & 數學模式中產生下標字 & \verb|\_{}| & \verb|\textunderscore| \\
\^{}           & 數學模式中產生上標字 & \verb|\^{}| & \verb|\textasciicircum| \\
\{             & 標示命令的作用範圍   & \verb|\{| & \verb|\textbraceleft|\\
\}             & 標示命令的作用範圍   & \verb|\}| & \verb|\textbraceright|\\
\textless      & 數學模式中的小於符號 & \verb|$<$| & \verb|\textless| \\
\textgreater & 數學模式中的大於符號   & \verb|$>$| & \verb|\textgreater| \\
\textbar     & OT1,數學模式中才能正確顯示 & \verb+$|$+ & \verb|\textbar| \\
\&           & 表格中的分隔符號   & \verb|\&| & NA \\
\hline
\end{tabular}
\end{center}
\vspace{.5\baselineskip}


  \marginpar{\back}

  \subsection{改變字型大小要用 \texttt{\textbackslash par} 來調整行距}

\def\mytext{%
《說文解字》書名。東漢許慎撰,三十卷,為我國第一部有系統分析字形及考究字源的字書。按文字形體及偏旁構造分列五百四十部,首創部首編排法?}

\begin{Wrong}
\begin{verbatim}
\begin{document}
\mytext

{\footnotesize \mytext}
\end{document}
\end{verbatim}
\mytext

{\footnotesize \mytext}
\end{Wrong}

\begin{Right}
\begin{verbatim}
\begin{document}
\mytext

{\footnotesize \mytext\par}
\end{document}
\end{verbatim}
\mytext

{\footnotesize \mytext\par}
\end{Right}
那個 \verb|\mytext| 是事先定義好的一段文字。在改變字型大小時要注意它的行距,它會依原先的行距來排版,要校正這個問題,要在 group 的 \verb|}| 之前先換成下一個段落。

不做這樣的調整的話,例子裡頭的小字(footnotesize)的段落,它的行距太大,因為是依原先normalsize的行距來排版(\TeX 是依段落來斷行的,在此之前一切資訊未定)。小字應依小字的比例來縮小行距,同理改變成大字時,也應依大字的行距來照比例調大。請參考\href{https://tex.stackexchange.com/questions/444039/why-do-i-have-to-use-par-if-i-change-font-size-withing-a-group-scope}{\textsf{StackExchange}}上的討論。

\marginpar{\back}


\section{套件使用}
\label{sec:package}

  \subsection{重複載入的套件}

\begin{Wrong}
\begin{verbatim}
\usepackage{hyperref}
\usepackage{url}
\end{verbatim}

\end{Wrong}

\begin{Right}
\begin{verbatim}
\usepackage{hyperref}
\end{verbatim}

\end{Right}

在\LaTeX 的使用上,引用套件(package)是避免不了,但\LaTeX 的套件,超過
四千個,這麼多的套件,其中難免會有衝突。很不幸的,並沒有很好的工具來預知
哪些套件會衝突,只能靠使用過的人的經驗及自行使用時的發現。

一些套件會預設載入其他套件,這樣這些預設會載入的套件就無需重複載入了。不
過,也很不幸的,並沒有一個完美的工具預知某套件會預設載入哪些其他的套件,
除非你打開這個套件的原始碼,去看看預設載入了什麼套件。或者加入一行 \verb|\listfiles| 於
其他套件載入之前,然後編譯後開啟 \verb|*.log| 檔,找到 \verb|*File List*| 的地方,
會列出所使用的套件及其版本\footnote{有一個很dirty的小程式{\tt ltxpkg},可以在\url{https://github.com/qtnez/luatexja-template/tree/main/tools}找到。}。

這個例子裡頭,{\sf hyperref}套件,預設就是會載入{\sf url}套件,因此無需
重複載入。那麼如果想傳參數給{\sf url}時怎麼辦?這時可以在{\sf hyperef}之前載入
{\sf url}並指定參數。

\begin{Code}
\setstretch{1.0}
\begin{verbatim}
\documentclass{article}
\usepackage[hyphens]{url}
\usepackage{hyperref}
\end{verbatim}
\end{Code}

\marginpar{\back}


\section{使用中文}
\label{sec:chinese}

\section{數理式子}
\label{sec:math}

  \subsection{數理式子需要適當的空白}

\begin{Wrong}
\begin{verbatim}
\[L(s)=\int^b_a\sqrt{1+(f'(x))^2}dx.\]
\end{verbatim}
\[L(s)=\int^b_a\sqrt{1+(f'(x))^2}dx.\]
\end{Wrong}

\begin{Right}
\begin{verbatim}
\[L(s)=\int^b_a \! \sqrt{1+(f'(x))^2}\,\text{d}x.\]
\end{verbatim}
\[L(s)=\int^b_a \! \sqrt{1+(f'(x))^2} \, \text{d}x.\]
\end{Right}

微分符號本身是一種運算子(operator),並不是變數,在數學模式\LaTeX 需要
數學斜體的變數來和其它字母做區分,他在此的地位類似單位,之前要留一個小空
白,而且必需使用正體(upright),不能使用數學斜體(italic)。這在\href{https://saso.gov.sa/ar/mediacenter/public_multimedia/Documents/SASO-ISO-800000-2-2020-E.pdf}{ISO 80000-2}標準裡頭也是如此認定。

但是這個微分符號是否要正體是有爭議的,大體而言,數學家較多偏向用數學
斜體(包括 Knuth 本人也是使用數學斜體),物理學家較多偏向
用正體\footnote{可以參考\href{https://tex.stackexchange.com/questions/14821/whats-the-proper-way-to-typeset-a-differential-operator}{\sf StackExchange}的討論。}。不過,既然有標準出現了,大家還是盡量遵循標準比較恰當\footnote{Knuth在設計\TeX 的時候,這個標準還沒有出現。}。

使用 \verb|\text{d}| 需要{\sf amsmath}套件,否則要使用 \LaTeX 內建的 \verb|\mathrm{d}|。另外{\sf physics}套件有提供 \verb|\dd, \dv| 的方便短指令。
另有一種取巧的方式,就是使用 \verb|\mbox{d}|,被 \verb|\mbox{}| 包住的文字都會使用
正體。這幾種方式會有小差異,\verb|\mbox{}| 的方式盡量避免,請試著編譯\\
\verb|$x^{\mathrm{a test}}x^{\mbox{a test}}x^{\text{a test}}x^{\textrm{a test}}$|\\
看出來的結果就知道了。

以下列出有關空白的指令:

\vspace{.5\baselineskip}
\begin{center}
\setstretch{1.2}
\begin{tabular}{llll}
\hline
指令 & 作用 & 指令 & 作用 \\
\hline
\verb|\quad| & 空出一個 em 單位的空白 & \verb|\qquad| & 空出兩個 em 的空白 \\
\verb|\,| & 加入 1/6 quad 的空白 & \verb|\!| & 減去 1/6 quad 的空白 \\
\verb|\;| & 加入 5/18 quad 的空白 & \verb|\:| & 加入 2/9 quad 的空白\\
\hline
\end{tabular}
\end{center}
\vspace{.5\baselineskip}

這裡要注意的地方是,自從\LaTeX\ 2020-10-01發行後,一些以往只能用在數學模式的
空白指令,現在已經可以用在文字模式及數學模式了。在此之前,
\verb|\,|、\verb|\quad| 及 \verb|\qquad| 可以用在一般的文字模式及數學模式,
其他的只能使用在數學模式中。

\marginpar{\back}

  \subsection{數學模式下使用正體字的情況}

\begin{Wrong}
\begin{verbatim}
$cos 2x=cos^2x-sin^2x$
\[lim_{n \to \infty}\sum_{i=1}^n{\frac{1}{n}}\]
\end{verbatim}
$cos 2x=cos^2x-sin^2x$
\[lim_{n \to \infty}\sum_{i=1}^n{\frac{1}{n}}\]
\end{Wrong}

\begin{Right}
\begin{verbatim}
$\cos 2x=\cos^2x-\sin^2x$
\[\lim_{n \to \infty}\sum_{i=1}^n{\frac{1}{n}}\]
\end{verbatim}
$\cos 2x=\cos^2x-\sin^2x$
\[\lim_{n \to \infty}\sum_{i=1}^n{\frac{1}{n}}\]
\end{Right}

數學模式下,為了容易區分字母所代表的意義,大原則是變數使用數學斜體。
但不是變數的情況,會有種種的慣例或標準來規範。

這裡整理一下{\sf ISO 80000-2}標準的相關說明:

\begin{itemize}
\item 變數(variables),例如$x$、$y$……等等。變動的數字(running numbers),例如$x_i$中的$i$。要用數學斜體(italic)。
\item 由於敘述上產生的函數,例如$f$、$g$,要用數學斜體。但已經是明確定義的固定函數,要用正體,例如$\sin$、$\exp$、$\ln$、$\lim$、$\log$……等等,要用正體。
\item 數學常數,例如$\text{e}=\num{2.718281828}\cdots$、$\uppi=\num{3.141592}\cdots$……等等,e 及$\uppi$要用正體。
\item 已經完整定義的運算子,例如微分符號($\text{d}x/\text{d}y$)、加減乘除以及數字,要用正體。
\end{itemize}

\marginpar{\back}


  \subsection{二種冒號(colon)}

\begin{Wrong}
\begin{verbatim}
\[\{\, x \colon x \notin x \,\}\]
\[f : x \to x^2\]
\end{verbatim}
\[\{\, x \colon x \notin x \,\}\]
\[f : x \to x^2\]
\end{Wrong}

\begin{Right}
\begin{verbatim}
\[\{\, x : x \notin x \,\}\]
\[f \colon x \to x^2\]
\end{verbatim}
\[\{\, x : x \notin x \,\}\]
\[f \colon x \to x^2\]
\end{Right}

二種(英文)冒號 \verb|:| 和 \verb|\colon| 表現出來,在形狀上雖然相同,但是置放位置不同。
通常 \verb|:| 是用在集合描述(關係運算符號),而 \verb|\colon| 是當成標點符號,
常用在映射表示\footnote{請參考\href{https://tex.stackexchange.com/questions/37789/using-colon-or-in-formulas}{\sf StackExchange}的討論。}。
另外,比例通常用 \verb|:|,例如$x:y:z = 3:4:5$。

\marginpar{\back}


\section{圖表處理}
\label{sec:figure}

\section{索引、文獻參考}
\label{sec:index}

\section{其他、雜項}
\label{sec:miscellaneous}

\clearpage
\hypertarget{contents}{}
\tableofcontents

\end{document}
