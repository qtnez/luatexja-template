%%%%%%%%%%%%%%%%%%%%%%%%%%%%%%%%%%%%%%%%%
% Frequently Asked Questions
% LaTeX Template
% Version 1.0 (22/7/13)
%
% This template has been downloaded from:
% http://www.LaTeXTemplates.com
%
% Original author:
% Adam Glesser (adamglesser@gmail.com)
%
% zh_TW(LuaLaTeX + luatex-ja) version by Edward G.J. Lee <edt1023@gmail.com>
%
% License:
% CC BY-NC-SA 3.0 (http://creativecommons.org/licenses/by-nc-sa/3.0/)
%
%%%%%%%%%%%%%%%%%%%%%%%%%%%%%%%%%%%%%%%%%

\documentclass[,a4paper,11pt]{article}

\usepackage[myfont]{ltj-zhfonts}
\renewcommand{\baselinestretch}{1.20}
\setlength{\marginparwidth }{2cm}

\usepackage[margin=1in]{geometry} % Required to make the margins smaller to fit more content on each page
\usepackage[linkcolor=blue]{hyperref} % Required to create hyperlinks to questions from elsewhere in the document
\hypersetup{pdfborder={0 0 0}, colorlinks=true, urlcolor=blue} % Specify a color for hyperlinks
\usepackage{todonotes} % Required for the boxes that questions appear in
\usepackage{tocloft} % Required to give customize the table of contents to display questions
%\usepackage{microtype} % Slightly tweak font spacing for aesthetics 中文文件不需要
%\usepackage{palatino} % Use the Palatino font
\usepackage{libertinus} % Use the libertinus font

\setlength\parindent{0pt} % Removes all indentation from paragraphs

% Create and define the list of questions
\newlistof{questions}{faq}{\large 經常被提問的問題列表} % This creates a new table of contents-like environment that will output a file with extension .faq
\setlength\cftbeforefaqtitleskip{4em} % Adjusts the vertical space between the title and subtitle
\setlength\cftafterfaqtitleskip{1em} % Adjusts the vertical space between the subtitle and the first question
\setlength\cftparskip{.3em} % Adjusts the vertical space between questions in the list of questions

% Create the command used for questions
\newcommand{\question}[1] % This is what you will use to create a new question
{
\refstepcounter{questions} % Increases the questions counter, this can be referenced anywhere with \thequestions
\par\noindent % Creates a new unindented paragraph
\phantomsection % Needed for hyperref compatibility with the \addcontensline command
\addcontentsline{faq}{questions}{#1} % Adds the question to the list of questions
\todo[inline,color=green!20]{\textbf{#1}} % Uses the todonotes package to create a fancy box to put the question
\vspace{1em} % White space after the question before the start of the answer
}

% Uncomment the line below to get rid of the trailing dots in the table of contents
%\renewcommand{\cftdot}{}

% Uncomment the two lines below to get rid of the numbers in the table of contents
%\let\Contentsline\contentsline
%\renewcommand\contentsline[3]{\Contentsline{#1}{#2}{}}

\begin{document}

%----------------------------------------------------------------------------------------
%	TITLE AND LIST OF QUESTIONS
%----------------------------------------------------------------------------------------

\begin{center}
\Huge{\bf \emph{FAQ 的範本}} % Main title
\end{center}

\listofquestions % This prints the subtitle and a list of all of your questions

\newpage % Comment this if you would like your questions and answers to start immediately after table of questions

%----------------------------------------------------------------------------------------
%	QUESTIONS AND ANSWERS
%----------------------------------------------------------------------------------------

\question{我如何增加新的問答條?}\label{new-question}

程式碼如下:

\begin{verbatim}
\question{新的提問內容。}\label{question-label}

新提問的答案。
\end{verbatim}

%------------------------------------------------

\question{為何新提問後有如下的標籤?\hyperref[new-question]{之前的提問}}\label{labels}

這不是必要的,但這樣可以連結至另一個不同的問題條,若需要要這樣的連結,就要寫成:

\begin{verbatim}
\hyperref[question-label]{之前的提問}
\end{verbatim}

\texttt{[question-label]}是標籤名(內部運作,不會顯示出來),\texttt{\{之前的提問\}}則是顯示出來的文字。\texttt{\\hyperref[question-lavel]}則是引用的方法。

%------------------------------------------------

\question{我如何修改主標題及次標題?}\label{change-title}

修改主標題,只要找到``TITLE AND LIST OF QUESTIONS''的區塊,把「FAQ 的範本」修改成你想的內容即可。修改次標題,找到如下的指令:

\begin{verbatim}
\newlistof{questions}{faq}{\large 經常被提問的問題列表}
\end{verbatim}

把的新內容取代過去就行了。

%------------------------------------------------

\question{如何修改問題列表的行距?}\label{change-spacing}

很簡單:

\begin{verbatim}
\setlength\cftparskip{.3em}
\end{verbatim}

修改\texttt{.3em}成你想要的行距。

%------------------------------------------------

\question{如果我想在問題列表中隱藏頁數或/且把尾隨問題之後的連續小點去除,要如何處理?}\label{page-numbering}

去除連續小點,請找到以下內容,把註解拿掉:

\begin{verbatim}
%\renewcommand{\cftdot}{}
\end{verbatim}
想把頁數也去除,請找到以下的內容,並把註解符號拿掉:
\begin{verbatim}
%\let\Contentsline\contentsline
%\renewcommand\contentsline[3]{\Contentsline{#1}{#2}{}}
\end{verbatim}

%------------------------------------------------

\question{提問的問題是否可以標號?}\label{number-questions}

可以,你可以用以下的指令得到目前問題的編號:

\begin{verbatim}
\thequestions
\end{verbatim}

例如,這個問題的標號是:\thequestions。你甚至可以把這個標號引入問題中:

\begin{verbatim}
問題 \thequestions:
\end{verbatim}

只要置放在你的問題內容之前即可(例如底下的「\textbf{問題  7:}」)。在文件前導區(preamble)\texttt{\textbackslash questions} 中,在 \texttt{\#1}之前置放以上的內容,文件將會自動產生問題編號。

%------------------------------------------------

\question{問題 \thequestions:我可以更改問題光棒的顏色嗎?}\label{question-color}

找到以下一行就可以更改顏色了(除了主顏色可以更改外,後面的數字愈大顏色愈深):

\begin{verbatim}
\todo[inline, color=green!20]{\textbf{#1}}
\end{verbatim}

%----------------------------------------------------------------------------------------

\end{document}
