%%%%%%%%%%%%%%%%%%%%%%%%%%%%%%%%%%%%%%%%%
% eBook
% LaTeX Template
% Version 1.0 (29/12/14)
%
% This template has been downloaded from:
% http://www.LaTeXTemplates.com
%
% Original author:
% Luis Cobo (luiscobogutierrez@gmail.com) with extensive modifications by:
% Vel (vel@latextemplates.com)
%
% 2021.05.01
% zh_TW(LuaLaTeX) version by Edward G.J. Lee <edt1023@gmail.com>
%
% License:
% CC BY-NC-SA 3.0 (http://creativecommons.org/licenses/by-nc-sa/3.0/)
%
%%%%%%%%%%%%%%%%%%%%%%%%%%%%%%%%%%%%%%%%%

%----------------------------------------------------------------------------------------
%	DOCUMENT CONFIGURATIONS AND INFORMATION
%----------------------------------------------------------------------------------------

\documentclass[oneside,11pt]{memoir} % Font size

% memoir 和 luatexja-fontspec 同時定義了 \printglossary,故需先取消。
\let\printglossary\relax

\input{structure.tex} % Include the file that specifies the document structure and layout

\title{格林童話故事} % Book title
\author{格林兄弟童話} % Author
\newcommand{\edition}{第二版} % Book edition

%----------------------------------------------------------------------------------------

\begin{document}

%----------------------------------------------------------------------------------------
%	TITLE PAGE
%----------------------------------------------------------------------------------------

\thispagestyle{empty} % Suppress page numbering
\ThisCenterWallPaper{1.12}{littlered.jpg} % Add the background image, the first argument is the scaling - adjust this as necessary so the image fits the entire page

\begin{tikzpicture}[remember picture,overlay]
\node[rectangle, rounded corners,fill=white, opacity=0.75,anchor=south west,
       minimum width=3cm, minimum height=6cm] (box) at (-0.5,-10) (box){};
% White rectangle - "minimum width/height" adjust the width and height of the box; "(-0.5,-10)" adjusts the position on the page
\node[anchor=west, color01, xshift=-1.5cm, yshift=-0.4cm, text width=2.9cm,
      font=\sffamily\scriptsize] at (box.north){\edition};
\node[anchor=west, color01, xshift=-1.5cm, yshift=-2cm, text width=2.9cm,
      font=\sffamily\bfseries\scshape\Large] at (box.north){\thetitle};
\node[anchor=west, color01, xshift=-1.5cm, yshift=-5cm, text width=2.9cm,
      font=\sffamily\bfseries] at (box.north){\theauthor};
% "Text width" adjusts the wrapping width, "xshift/yshift" adjust the position relative to the white rectangle
\end{tikzpicture}

\newpage % Make sure the following content is on a new page

%----------------------------------------------------------------------------------------
%	TABLE OF CONTENTS
%----------------------------------------------------------------------------------------

\tableofcontents % Prints the table of contents

%----------------------------------------------------------------------------------------
%	INTRODUCTION SECTION
%----------------------------------------------------------------------------------------

\chapter*{簡介/引子} % Introduction chapter suppressed from the table of contents

\begin{quote}
This is one of my finer quotations.\\
--John Smith
\end{quote}

這裡是最好書寫書籍引子的地方\footnote{你甚至可以使用腳註,顯得更有智慧。}。

%----------------------------------------------------------------------------------------
%	BOOK PART
%----------------------------------------------------------------------------------------

\part{童話故事}

%----------------------------------------------------------------------------------------
%	CHAPTER ONE
%----------------------------------------------------------------------------------------

\chapter{小紅帽}

從前有個可愛的小姑娘,誰見了都喜歡,但最喜歡她的是她的奶奶,簡直是她要什
麼就給她什麼。一次,奶奶送給小姑娘一頂用絲絨做的小紅帽,戴在她的頭上正好
合適。從此,姑娘再也不願意戴任何別的帽子,於是大家便叫她「小紅帽」。

一天,媽媽對小紅帽說:「來,小紅帽,這裡有一塊蛋糕和一瓶葡萄酒,快給奶奶
送去,奶奶生病了,身子很虛弱,吃了這些就會好一些的。趁著現在天還沒有熱,
趕緊動身吧。在路上要好好走,不要跑,也不要離開大路,否則你會摔跤的,那樣
奶奶就什麼也吃不上了。到奶奶家的時候,別忘了說『早上好』,也不要一進屋就
東瞧西瞅。」

「我會小心的。」小紅帽對媽媽說,並且還和媽媽拉手作保證。

奶奶住在村子外面的森林裡,離小紅帽家有很長一段路。小紅帽剛走進森林就碰到
了一條狼。小紅帽不知道狼是壞傢伙,所以一點也不怕它。

「你好,小紅帽,」狼說。

「謝謝你,狼先生。」

「小紅帽,這麼早要到哪裡去呀?」

「我要到奶奶家去。」

「你那圍裙下面有什麼呀?」

「蛋糕和葡萄酒。昨天我們家烤了一些蛋糕,可憐的奶奶生了病,要吃一些好東西
才能恢復過來。」

「你奶奶住在哪裡呀,小紅帽?」

「進了林子還有一段路呢。她的房子就在三棵大橡樹下,低處圍著核桃樹籬笆。你
一定知道的。」小紅帽說。

狼在心中盤算著:「這小東西細皮嫩肉的,味道肯定比那老太婆要好。我要講究一
下策略,讓她倆都逃不出我的手心。」於是它陪著小紅帽走了一會兒,然後說:「
小紅帽,你看周圍這些花多麼美麗啊!幹嗎不回頭看一看呢?還有這些小鳥,它們
唱得多麼動聽啊!你大概根本沒有聽到吧?林子裡的一切多麼美好啊,而你卻只管
往前走,就像是去上學一樣。」

小紅帽抬起頭來,看到陽光在樹木間來回跳蕩,美麗的鮮花在四周開放,便想:「
也許我該摘一把鮮花給奶奶,讓她高興高興。現在天色還早,我不會去遲的。」她
於是離開大路,走進林子去採花。她每採下一朵花,總覺得前面還有更美麗的花朵
,便又向前走去,結果一直走到了林子深處。

就在此時,狼卻直接跑到奶奶家,敲了敲門。

「是誰呀?」

「是小紅帽。」狼回答,「我給你送蛋糕和葡萄酒來了。快開門哪。」

「你拉一下門栓就行了,」奶奶大聲說,「我身上沒有力氣,起不來。」

狼剛拉起門栓,那門就開了。狼二話沒說就衝到奶奶的床前,把奶奶吞進了肚子。
然後她穿上奶奶的衣服,戴上她的帽子,躺在床上,還拉上了簾子。

可這時小紅帽還在跑來跑去地採花。直到採了許多許多,她都拿不了啦,她才想起
奶奶,重新上路去奶奶家。

看到奶奶家的屋門敞開著,她感到很奇怪。她一走進屋子就有一種異樣的感覺,心
中便想:「天哪!平常我那麼喜歡來奶奶家,今天怎麼這樣害怕?」她大聲叫道:
「早上好!」,可是沒有聽到回答。她走到床前拉開簾子,只見奶奶躺在床上,帽
子拉得低低的,把臉都遮住了,樣子非常奇怪。

「哎,奶奶,」她說,「你的耳朵怎麼這樣大呀?」

「為了更好地聽你說話呀,乖乖。」

「可是奶奶,你的眼睛怎麼這樣大呀?」小紅帽又問。

「為了更清楚地看你呀,乖乖。」

「奶奶,你的手怎麼這樣大呀?」

「可以更好地抱著你呀。」

「奶奶,你的嘴巴怎麼大得很嚇人呀?」

「可以一口把你吃掉呀!」

狼剛把話說完,就從床上跳起來,把小紅帽吞進了肚子,狼滿足了食慾之後便重新
躺到床上睡覺,而且鼾聲震天。一位獵人碰巧從屋前走過,心想:「這老太太鼾打
得好響啊!我要進去看看她是不是出什麼事了。」獵人進了屋,來到床前時卻發現
躺在那裡的竟是狼。「你這老壞蛋,我找了你這麼久,真沒想到在這裡找到你!」
他說。他正準備向狼開槍,突然又想到,這狼很可能把奶奶吞進了肚子,奶奶也許
還活著。獵人就沒有開槍,而是操起一把剪刀,動手把呼呼大睡的狼的肚子剪了開
來。他剛剪了兩下,就看到了紅色的小帽子。他又剪了兩下,小姑娘便跳了出來,
叫道:「真把我嚇壞了!狼肚子裡黑漆漆的。」接著,奶奶也活著出來了,只是有
點喘不過氣來。小紅帽趕緊跑去搬來幾塊大石頭,塞進狼的肚子。狼醒來之後想逃
走,可是那些石頭太重了,它剛站起來就跌到在地,摔死了。

三個人高興極了。獵人剝下狼皮,回家去了;奶奶吃了小紅帽帶來的蛋糕和葡萄酒
,精神好多了;而小紅帽卻在想:「要是媽媽不允許,我一輩子也不獨自離開大路
,跑進森林了。」

人們還說,小紅帽後來又有一次把蛋糕送給奶奶,而且在路上又有一隻狼跟她搭話
,想騙她離開大路。可小紅帽這次提高了警惕,頭也不回地向前走。她告訴奶奶她
碰到了狼,那傢伙嘴上雖然對她說「你好」,眼睛裡卻露著兇光,要不是在大路上
,它準把她給吃了。「那麼,」奶奶說,「我們把門關緊,不讓它進來。」不一會
兒,狼真的一面敲著門一面叫道:「奶奶,快開門呀。我是小紅帽,給你送蛋糕來
了。」但是她們既不說話,也不開門。這長著灰毛的傢伙圍著房子轉了兩三圈,最
後跳上屋頂,打算等小紅帽在傍晚回家時偷偷跟在她的後面,趁天黑把她吃掉。可
奶奶看穿了這傢伙的壞心思。她想起屋子前有一個大石頭槽子,便對小姑娘說:「
小紅帽,把桶拿來。我昨天做了一些香腸,提些煮香腸的水去倒進石頭槽裡。」小
紅帽提了很多很多水,把那個大石頭槽子裝得滿滿的。香腸的氣味飄進了狼的鼻孔
,它使勁地用鼻子聞呀聞,並且朝下張望著,到最後把脖子伸得太長了,身子開始
往下滑。它從屋頂上滑了下來,正好落在大石槽中,淹死了。小紅帽高高興興地回
了家,從此再也沒有誰傷害過她。

%----------------------------------------------------------------------------------------
%	CHAPTER TWO
%----------------------------------------------------------------------------------------

\chapter{糖果屋(漢賽爾與格萊特)}

在大森林的邊上,住著一個貧窮的樵夫,他妻子和兩個孩子與他相依為命。他的兒
子名叫漢賽爾,女兒名叫格萊特。他們家裡原本就缺吃少喝,而這一年正好遇上國
內物價飛漲,樵夫一家更是吃了上頓沒下頓,連每天的麵包也無法保證。這天夜裡
,愁得輾轉難眠的樵夫躺在床上大傷腦筋,他又是歎氣,又是呻吟。終於他對妻子
說:「咱們怎麼辦哪!自己都沒有一點吃的,又拿什麼去養咱們那可憐的孩子啊?
」

「聽我說,孩子他爹,」他老婆回答道:「明天大清早咱們就把孩子們帶到遠遠的
密林中去,在那兒給他們生一堆火,再給他們每人一小塊麵包,然後咱們就假裝去
幹咱們的活,把他們單獨留在那兒。他們不認識路,回不了家,咱們就不用再養他
們啦。」

「不行啊,老婆,」樵夫說:「我不能這麼幹啊。我怎麼忍心把我的孩子丟在叢林
裡喂野獸呢!」

「哎,你這個笨蛋,」他老婆說,「不這樣的話,咱們四個全都得餓死!」接著她
又嘰哩呱啦、沒完沒了地勸他,最後,他也就只好默許了。

那時兩個孩子正餓得無法入睡,正好聽見了繼母與父親的全部對話。聽見繼母對父
親的建議,格萊特傷心地哭了起來,對漢賽爾說:「這下咱倆可全完了。」

「別吱聲,格萊特,」漢賽爾安慰她說,「放心吧,我會有辦法的。」

等兩個大人睡熟後,他便穿上小外衣,開啟後門偷偷溜到了房外。這時月色正明,
皎潔的月光照得房前空地上的那些白色小石子閃閃發光,就像是一塊塊銀幣。漢賽
爾蹲下身,盡力在外衣口袋裡塞滿白石子。然後他回屋對格萊特說:「放心吧,小
妹,只管好好睡覺就是了,上帝會與我們同在的。」

說完,他回到了他的小床上睡覺。

天剛破曉,太陽還未躍出地平線,那個女人就叫醒了兩個孩子,「快起來,快起來
,你們這兩個懶蟲!」她嚷道,「我們要進山砍柴去了。」說著,她給一個孩子一
小塊麵包,並告誡他們說:「這是你們的午飯,可別提前吃掉了,因為你們再也甭
想得到任何東西了。」格萊特接過麵包藏在她的圍裙底下,因為漢賽爾的口袋裡這
時塞滿了白石子。

隨後,他們全家就朝著森林進發了。漢賽爾總是走一會兒便停下來回頭看看自己的
家,走一會兒便停下來回頭看自己的家。他的父親見了便說:「漢賽爾,你老是回
頭瞅什麼?專心走你的路。」

「哦,爸爸,」漢賽爾回答說:「我在看我的白貓呢,他高高地蹲在屋頂上,想跟
我說『再見』呢!」

「那不是你的小貓,小笨蛋,」繼母講,「那是早晨的陽光照在煙囪上。」其實漢
賽爾並不是真的在看小貓,他是悄悄地把亮亮的白石子從口袋裡掏出來,一粒一粒
地丟在走過的路上。

到了森林的深處,他們的父親對他們說:「嗨,孩子們,去拾些柴火來,我給你們
生一堆火。」

漢賽爾和格萊特拾來許多枯枝,把它們堆得像小山一樣高。當枯枝點著了,火焰升
得老高後,繼母就對他們說:「你們兩個躺到火堆邊上去吧,好好呆著,我和你爸
爸到林子裡砍柴。等一干完活,我們就來接你們回家。」

於是漢賽爾和格萊特坐在火堆旁邊,等他們的父母幹完活再來接他們。到了中午時
分,他們就吃掉了自己的那一小塊麵包。因為一直能聽見斧子砍樹的彭、彭聲,他
們相信自己的父親就在近旁。其實他們聽見的根本就不是斧子發出的聲音,那是一
根綁在一棵小樹上的枯枝,在風的吹動下撞在樹幹上發出來的聲音。兄妹倆坐了好
久好久,疲倦得上眼皮和下眼皮都打起架來了。沒多久,他們倆就呼呼睡著了,等
他們從夢中醒來時,已是漆黑的夜晚。格萊特害怕得哭了起來,說:「這下咱們找
不到出森林的路了!」

「別著急,」漢賽爾安慰她說,「等一會兒月亮出來了,咱們很快就會找到出森林
的路。」

不久,當一輪滿月升起來時,漢賽爾就拉著他妹妹的手,循著那些月光下像銀幣一
樣在地上閃閃發光的白石子指引的路往前走。他們走了整整的一夜,在天剛破曉的
時候回到了他們父親的家門口。他們敲敲門,來開門的是他們的繼母。她開啟門一
見是漢賽爾和格萊特,就說:「你們怎麼在森林裡睡了這麼久,我們還以為你們不
想回家了吶!」

看到孩子,父親喜出望外,因為冷酷地拋棄兩個孩子,他心中十分難受。

他們一家又在一起艱難地生活了。但時隔不久,又發生了全國性的饑荒。一天夜裡
,兩個孩子又聽見繼母對他們的父親說:「哎呀!能吃的都吃光了,就剩這半個麵
包,你看以後可怎麼辦啊?咱們還是得減輕負擔,必須把兩個孩子給扔了!這次咱
們可以把他們帶進更深、更遠的森林中去,叫他們再也找不到路回來。只有這樣才
能挽救我們自己。」

聽見妻子又說要拋棄孩子,樵夫心裡十分難過。他心想,大家同甘共苦,共同分享
最後一塊麵包不是更好嗎?但是像天下所有的男人一樣,對一個女人說個「不」字
那是太難太難了,樵夫也毫不例外。就像是「誰套上了籠頭,誰就必須得拉車」的
道理一樣,樵夫既然對妻子作過第一次讓步,當然就必然有第二次讓步了,他也就
不再反對妻子的建議了。

然而,孩子們聽到了他們的全部談話。等父母都睡著後,漢賽爾又從床上爬了起來
,想溜出門去,像上次那樣,到外邊去撿些小石子,但是這次他發現門讓繼母給鎖
死了。但他心裡又有了新的主意,他又安慰他的小妹妹說:「別哭,格萊特,不用
擔心,好好睡覺。上帝會幫助咱們的。」

一大清早,繼母就把孩子們從床上揪了下來。她給了他們每人一塊麵包,可是比上
次那塊要小多了。

在去森林的途中,漢賽爾在口袋裡捏碎了他的麵包,並不時地停下腳步,把碎麵包
屑撒在路上。

「漢賽爾,你磨磨蹭蹭地在後面看什麼?」他的父親見他老是落在後面就問他。「
我在看我的小鴿子,它正站在屋頂上『咕咕咕』地跟我說再見呢。」漢賽爾回答說
。

「你這個白痴,」他繼母叫道,「那不是你的鴿子,那是早晨的陽光照在煙囪上面
。」但是漢賽爾還是在路上一點一點地撒下了他的麵包屑。

繼母領著他們走了很久很久,來到了一個他們從未到過的森林中。像上次一樣,又
生起了一大堆火,繼母又對他們說:「好好呆在這兒,孩子們,要是困了就睡一覺
,我們要到遠點的地方去砍柴,幹完活我們就來接你們。」

到了中午,格萊特把她的麵包與漢賽爾分來吃了,因為漢賽爾的麵包已經撒在路上
了。然後,他們倆又睡著了。一直到了半夜,仍然沒有人來接這兩個可憐的孩子,
他們醒來已是一片漆黑。漢賽爾安慰他的妹妹說:「等月亮一出來,我們就看得見
我撒在地上的麵包屑了,它一定會指給我們回家的路。」

但是當月亮升起來時,他們在地上卻怎麼也找不到一點麵包屑了,原來它們都被那
些在樹林裡、田野上飛來飛去的鳥兒一點點地啄食了。

雖然漢賽爾也有些著急了,但他還是安慰妹妹說:「我們一定能找到路的,格萊特。」

但他們沒有能夠找到路,雖然他們走了一天一夜,可就是出不了森林。他們已經餓
得頭昏眼花,因為除了從地上找到的幾顆草黴,他們沒吃什麼東西。這時他們累得
連腳都邁不動了,倒在一顆樹下就睡著了。

這已是他們離開父親家的第三天早晨了,他們深陷叢林,已經迷路了。如果再不能
得到幫助,他們必死無疑。就在這時,他們看到了一隻通體雪白的、極其美麗的鳥
兒站在一根樹枝上引吭高歌,它唱得動聽極了,他們兄妹倆不由自主地停了下來,
聽它唱。它唱完了歌,就張開翅膀,飛到了他們的面前,好像示意他們跟它走。他
們於是就跟著它往前走,一直走到了一幢小屋的前面,小鳥停到小屋的房頂上。他
倆這時才發現小屋居然是用香噴噴的麵包做的,房頂上是厚厚的蛋糕,窗戶卻是明
亮的糖塊。

「讓我們放開肚皮吧,」漢賽爾說:「這下我們該美美地吃上一頓了。我要吃一小
塊房頂,格萊特,你可以吃窗戶,它的味道肯定美極了、甜極了。」

說著,漢賽爾爬上去掰了一小塊房頂下來,嘗著味道。格萊特卻站在窗前,用嘴去
啃那個甜窗戶。這時,突然從屋子裡傳出一個聲音:

「啃啊!啃啊!啃啊啃!誰在啃我的小房子?」

孩子們回答道:

「是風啊,是風,是天堂裡的小娃娃。」

他們邊吃邊回答,一點也不受幹擾。

漢賽爾覺得房頂的味道特別美,便又拆下一大塊來;格萊特也乾脆摳下一扇小圓窗
,坐在地上慢慢享用。突然,房子的門開啟了,一個老婆婆拄著柺杖顫顫巍巍的走
了出來。漢賽爾和格萊特嚇得雙腿打顫,拿在手裡的食物也掉到了地上。

那個老婆婆晃著她顫顫巍巍的頭說:「好孩子,是誰帶你們到這兒來的?來,跟我
進屋去吧,這兒沒人會傷害你們!」

她說著就拉著兄妹倆的手,把他們領進了她的小屋,並給他們準備了一頓豐盛的晚
餐,有牛奶、糖餅、蘋果,還有堅果。等孩子們吃完了,她又給孩子們舖了兩張白
色的小床,漢賽爾和格萊特往床上一躺,馬上覺得是進了天堂。

其實這個老婆婆是笑裡藏刀,她的友善只是偽裝給他們看的,她事實上是一個專門
引誘孩子上當的邪惡的巫婆,她那幢用美食建造的房子就是為了讓孩子們落入她的
圈套。一旦哪個孩子落入她的魔掌,她就殺死他,把他煮來吃掉。這個巫婆的紅眼
睛視力不好,看不遠,但是她的嗅覺卻像野獸一樣靈敏,老遠老遠她就能嗅到人的
味道。漢賽爾和格萊特剛剛走近她的房子她就知道了,高興得一陣狂笑,然後就冷
笑著打定了主意:「我要牢牢地抓住他們,決不讓他們跑掉。」

第二天一早,還不等孩子們醒來,她就起床了。看著兩個小傢夥那紅撲撲、圓滾滾
的臉蛋,她忍不住口水直流:「好一頓美餐吶!」說著便抓住漢賽爾的小胳膊,把
他扛進了一間小馬廄,並用柵欄把他鎖了起來。漢賽爾在裡面大喊大叫,可是毫無
用處。然後,老巫婆走過去把格萊特搖醒,沖著她吼道:「起來,懶丫頭!快去打
水來替你哥哥煮點好吃的。他關在外面的馬廄裡,我要把他養得白白胖胖的,然後
吃掉他。」

格萊特聽了傷心得大哭起來,可她還是不得不按照那個老巫婆的吩咐去幹活。於是
,漢賽爾每天都能吃到許多好吃的,而可憐的格萊特每天卻只有螃蟹殼吃。每天早
晨,老巫婆都要顫顫巍巍的走到小馬廄去喊漢賽爾:「漢賽爾,把你的手指頭伸出
來,讓我摸摸你長胖了沒有!」可是漢賽爾每次都是伸給她一根啃過的小骨頭,老
眼昏花的老巫婆,根本就看不清楚,她還真以為是漢賽爾的手指頭呢!她心裡感到
非常納悶,怎麼漢賽爾還沒有長胖一點呢?

又過了四個星期,漢賽爾還是很瘦的樣子。老巫婆失去了耐心,便揚言她不想再等了。

「過來,格萊特,」她對小女孩吼道,「快點去打水來!管他是胖還是瘦,明天我
一定要殺死漢賽爾,把他煮來吃了。」

可憐的小妹妹被逼著去打水來準備煮她的哥哥,一路上她傷心萬分,眼淚順著臉頰
一串一串地往下掉!「親愛的上帝,請幫幫我們吧!」她呼喊道,「還不如當初在
森林裡就被野獸吃掉,那我們總還是死在一起的呵!」

趁老巫婆離開一會兒,可憐的格萊特瞅準機會跑到漢賽爾身邊,把她所聽到的一切
都告訴他:

「我們要趕快逃跑,因為這個老太婆是個邪惡的巫婆,她要殺死我們哩。」

可是漢賽爾說:「我知道怎麼逃出去,因為我已經把插銷給搞開了。不過,你得首
先去把她的魔杖和掛在她房間裡的那根笛子偷來,這樣萬一她追來,我們就不怕她
了。」

等格萊特好不容易把魔杖和笛子都偷來之後,兩個孩子便逃跑了。

這時,老巫婆走過來看她的美餐是否弄好了,發現兩個孩子卻不見了。雖說她的眼
睛不好,可她還是從視窗看到了那兩個正在逃跑的孩子。

她勃然大怒,趕緊穿上她那雙一步就能走上幾碼遠的靴子,不多一會就要趕上那兩
個孩子了。格萊特眼看老巫婆就要追上他們了,便用她偷來的那根魔杖把漢賽爾變
成了一個湖泊,而把她自己變成了一隻在湖泊中游來遊去的小天鵝。老巫婆來到湖
邊,往湖裡扔了些面包屑想騙那隻小天鵝上當。可是小天鵝就是不過來,最後老巫
婆只好空著手回去了。

見到老巫婆走了,格萊特便用那根魔杖又把自己和漢賽爾變回了原來的模樣。然後
,他們又繼續趕路,一直走到天黑。

很快,老巫婆又追了上來。

這時,小姑娘把自己變成了山楂樹籬笆中的一朵玫瑰,於是漢賽爾便在這只玫瑰的
旁邊坐了下來變成一位笛手。

「吹笛子的好心人,」老巫婆說,「我可以摘下那朵漂亮的玫瑰花嗎?」

「哦,可以。」漢賽爾說。

於是,非常清楚那朵玫瑰是什麼的老巫婆快步走向樹籬想飛快地摘下它。就在這時
,漢賽爾拿出他的笛子,吹了起來。

這是一根魔笛,誰聽了這笛聲都會不由自主地跳起舞來。所以那老巫婆不得不隨著
笛聲一直不停地旋轉起來,再也摘不到那朵玫瑰了。漢賽爾就這樣不停地吹著,直
吹到那些荊棘把巫婆的衣服掛破,並深深地刺到她的肉裡,直刺得她哇哇亂叫。最
後,老巫婆被那些荊棘給牢牢地纏住了。

這時,格萊特又恢復了自己的原形,和漢賽爾一塊兒往家走去。走了長長的一段路
程之後,格萊特累壞了。於是他們便在靠近森林的草地上找到了一棵空心樹,就在
樹洞裡躺了下來。就在他們睡著的時候,那個好不容易從荊棘叢中脫身出來的老巫
婆又追了上來。她一看到自己的魔杖,就得意地一把抓住它。然後,立刻把可憐的
漢賽爾變成了一頭小鹿。

格萊特醒來之後,看到所發生的一切,傷心地撲到那頭可憐的小動物身上哭了起來
。這時,淚水也從小鹿的眼睛裡不停地往下流。

格萊特說:「放心吧,親愛的小鹿,我絕不會離開你。」

說著,她就取下她那長長的金色項鍊戴到他的脖子上,然後又扯下一些燈芯草把它
編成一條草繩,套住小鹿的脖子,無論她走到哪兒,她都把這頭可憐的小鹿帶在身
邊。

終於,有一天他們來到了一個小屋前。格萊特看到這間小屋沒有人住,便說:「我
們就在這兒住下吧。」

她採來了很多樹葉和青苔替小鹿舖了一張柔軟的小床。每天早上,她便出去採摘一
些堅果和漿果來充饑,又替她的哥哥採來很多樹葉和青草。她把樹葉和青草放在自
己的手中喂小鹿,而那頭小鹿就在她的身旁歡快地蹦來蹦去。到了晚上,格萊特累
了,就會把頭枕在小鹿的身上睡覺。要是可憐的漢賽爾能夠恢復原形,那他們的生
活該有多幸福啊!

他們就這樣在森林裡生活了許多年,這時,格萊特已經長成了一個少女。有一天,
剛好國王到這兒來打獵。當小鹿聽到在森林中迴盪的號角聲、獵狗汪汪的叫聲以及
獵人們的大喊聲時,忍不住想去看看是怎麼回事。「哦,妹妹,」他說,「讓我到
森林裡去看看吧,我再也不能待在這兒了。」他不斷地懇求著,最後她只好同意讓
他去了。

「可是,」她說,「一定要在天黑之前回來。我會把門關好不讓那些獵人們進來。
如果你敲門並說:『妹妹,讓我進來。』我就知道是你回來了。如果你不說話,我
就把門緊緊地關住。」

於是小鹿便一蹦一跳地跑了出去。當國王和他的獵人們看到這頭美麗的小鹿之後,
便來追趕他,可是他們怎麼也逮不著他,因為當他們每次認為自己快要抓住他時,
他都會跳到樹叢中藏起來。

天黑了下來,小鹿便跑回了小屋,他敲了敲門說:「妹妹,讓我進來吧!」於是格
萊特便開啟了門,他跳了進來,在他那溫軟的床上美美地睡了一覺。

第二天早上,圍獵又開始了。小鹿一聽到獵人們的號角聲,他便說:「妹妹,替我
把門開啟吧。我一定要出去。」

國王和他的獵人們見到這頭小鹿,馬上又開始了圍捕。他們追了他一整天,最後終
於把他給圍住了,其中一個獵人還射中了他的一條腳。他一瘸一拐地好不容易才逃
回到了家中。那個射傷了他的獵人跟蹤著他,聽到了這頭小鹿說:「妹妹,讓我進
來吧。」還看到了那扇門開了,小鹿進去後很快又關上了。於是這個獵人就回去向
國王稟報了他的所見所聞。國王說:「那明天我們再圍捕一次吧。」

當格萊特看到她那親愛的小鹿受傷了,感到非常害怕。不過,她還是替他把傷口清
洗得乾乾淨淨,敷上了一些草藥。第二天早上,那傷口竟已經復原了。當號角再次
吹響的時候,那小東西又說:「我不能待在這兒,我必須出去看看。我會多加小心
,不會讓他們抓住我的。」

可是格萊特說:「我肯定他們這一次會殺死你的,我不讓你去。」

「如果你把我關在這兒的話,那我會遺憾而死。」他說。格萊特不得不讓他出去,
她心情沉重地開啟門,小鹿便又歡快地向林中奔去。

國王一看到小鹿,便大聲下令:「你們今天一定要追到他,可你們誰也不許傷害他。」

然而,太陽落山的時候,他們還是沒能抓住他。於是國王對那個曾經跟蹤過小鹿的
獵人說:「那麼現在領我去那個小屋吧。」

於是他們來到了小屋前,國王敲了敲門,並且說:「妹妹,讓我進來吧。」

門兒開啟之後,國王走了進去,只見房子裡站著一個他生平見過的最美麗的少女。

當格萊特看到來者並非是她的小鹿而是一位戴著皇冠的國王時,感到非常害怕。可
是國王非常友善地拉著她的手,並說:「你願意和我一起到我的城堡去,做我的妻
子嗎?」

「是的,」格萊特說,「我可以和你一起去你的城堡,可是我不能成為你的妻子,
因為我的小鹿必須和我在一起,我不能和他分開。」

「那好吧,」國王說,「他可以和你一起去,永遠都不離開你,並且他想要什麼就
會有什麼。」

正在這時,小鹿跳了進來。於是格萊特把草繩套在他的脖子上,他們便一起離開了小屋。

國王把小格萊特抱上他的高頭大馬之後,就朝著他的王宮跑去。那頭小鹿也歡快地
跟在他們後面。一路上,格萊特告訴了國王有關她的一切,國王認識那個老巫婆,
便派人去把她叫來,命令她恢復小鹿的人形。

當格萊特看到他親愛的哥哥又恢復了原形,她非常感激國王,便欣然同意嫁給他。
他們就這樣幸福地生活著,漢賽爾也成了國王的王宮大臣。格林童話英文版:漢賽
爾與格萊特

%----------------------------------------------------------------------------------------
%	CHAPTER THREE
%----------------------------------------------------------------------------------------

\chapter{長髮姑娘}

從前有一個男人和一個女人,他倆一直想要個孩子,可總也得不到。最後,女人只
好希望上帝能賜給她一個孩子。他們家的屋子後面有個小窗戶,從那裡可以看到一
個美麗的花園,裡面長滿了奇花異草。可是,花園的周圍有一道高牆,誰也不敢進
去,因為那個花園屬於一個女巫。這個女巫的法力非常大,世界上人人都怕她。一
天,妻子站在視窗向花園望去,看到一塊菜地上長著非常漂亮的萵苣。這些萵苣綠
油油、水靈靈的,立刻就勾起了她的食慾,非常想吃它們。這種慾望與日俱增,而
當知道自己無論如何也吃不到的時候,她變得非常憔悴,臉色蒼白,痛苦不堪。她
丈夫嚇壞了,問她:「親愛的,你哪裡不舒服呀?」「啊,」她回答,「我要是吃
不到我們家後面那個園子裡的萵苣,我就會死掉的。」丈夫因為非常愛她,便想:
「與其說讓妻子去死,不如給她弄些萵苣來,管它會發生什麼事情呢。」黃昏時分
,他翻過圍牆,溜進了女巫的花園,飛快地拔了一把萵苣,帶回來給她妻子吃。妻
子立刻把萵苣做成色拉,狼吞虎嚥地吃了下去。這萵苣的味道真是太好了,第二天
她想吃的萵苣居然比前一天多了兩倍。為了滿足妻子,丈夫只好決定再次翻進女巫
的園子。於是,黃昏時分,他偷偷地溜進了園子,可他剛從牆上爬下來,就嚇了一
跳,因為他看到女巫就站在他的面前。「你好大的膽子,」她怒氣衝衝地說,「竟
敢溜進我的園子來,像個賊一樣偷我的萵苣!」「唉,」他回答,「可憐可憐我,
饒了我吧。我是沒辦法才這樣做的。我妻子從視窗看到了你園子中的萵苣,想吃得
要命,吃不到就會死掉的。」女巫聽了之後氣慢慢消了一些,對他說:「如果事情
真像你說的這樣,我可以讓你隨便採多少萵苣,但我有一個條件:你必須把你妻子
將要生的孩子交給我。我會讓她過得很好的,而且會像媽媽一樣對待她。」丈夫由
於害怕,只好答應女巫的一切條件。妻子剛剛生下孩子,女巫就來了,給孩子取了
個名字叫「萵苣」,然後就把孩子帶走了。

「萵苣」慢慢長成了天底下最漂亮的女孩。孩子十二歲那年,女巫把她關進了一座
高塔。這座高塔在森林裡,既沒有樓梯也沒有門,只是在塔頂上有一個小小的窗戶
。每當女巫想進去,她就站在塔下叫道:

「萵苣,萵苣,把你的頭髮垂下來。」

萵苣姑娘長著一頭金絲般濃密的長髮。一聽到女巫的叫聲,她便鬆開她的髮辮,把
頂端繞在一個窗鉤上,然後放下來二十公尺。女巫便順著這長髮爬上去。

一兩年過去了。有一天,王子騎馬路過森林,剛好經過這座塔。這時,他突然聽到
美妙的歌聲,不由得停下來靜靜地聽著。唱歌的正是萵苣姑娘,她在寂寞中只好
靠唱歌來打發時光。王子想爬到塔頂上去見她,便四處找門,可怎麼也沒有找到
。他回到了宮中,那歌聲已經深深地打動了他,他每天都要騎馬去森林裡聽。一
天,他站在一棵樹後,看到女巫來了,而且聽到她衝著塔頂叫道:

「萵苣,萵苣,把你的頭髮垂下來。」

萵苣姑娘立刻垂下她的髮辮,女巫順著它爬了上去。王子想:「如果那就是讓人爬
上去的梯子,我也可以試試我的運氣。」第二天傍晚,他來到塔下叫道:

「萵苣,萵苣,把你的頭髮垂下來。」

頭髮立刻垂了下來,王子便順著爬了上去。

萵苣姑娘看到爬上來的是一個男人時,真的大吃一驚,因為她還從來沒有看到過男
人。但是王子和藹地跟她說話,說他的心如何如何被她的歌聲打動,一刻也得不
到安寧,非要來見她。萵苣姑娘慢慢地不再感到害怕,而當他問她願不願意嫁給
他時,她見王子又年輕又英俊,便想:「這個人肯定會比那教母更喜歡我。」她
於是就答應了,並把手伸給王子。她說:「我非常願意跟你一起走,可我不知道
怎麼下去。你每次來的時候都給我帶一根絲線吧,我要用絲線編一個梯子。等到梯
子編好了,我就爬下來,你就把我抱到你的馬背上。」因為老女巫總是在白天來
,所以他倆商定讓王子每天傍晚時來。女巫什麼也沒有發現,直到有一天萵苣姑
娘問她:「我問你,教母,我拉你的時候怎麼總覺得你比那個年輕的王子重得多
?他可是一下子就上來了。」「啊!你這壞孩子!」女巫嚷道,「你在說什麼?
我還以為你與世隔絕了呢,卻不想你竟然騙了我!」她怒氣衝衝地一把抓住萵苣姑
娘漂亮的辮子,在左手上纏了兩道,又用右手操起一把剪刀,喳喳喳幾下,美麗
的辮子便落在了地上。然後,她又狠心地把萵苣姑娘送到一片荒野中,讓她悽慘
痛苦地生活在那裡。

萵苣姑娘被送走的當天,女巫把剪下來的辮子綁在塔頂的窗鉤上。王子走來喊道:

「萵苣,萵苣,把你的頭髮垂下來。」

女巫放下頭髮,王子便順著爬了上去。然而,他沒有見到心愛的萵苣姑娘,卻看到
女巫正惡狠狠地瞪著他。「啊哈!」她嘲弄王子說,「你是來接你的心上人的吧
?可美麗的鳥兒不會再在窩裡唱歌了。她被貓抓走了,而且貓還要把你的眼睛挖
出來。你的萵苣姑娘完蛋了,你別想再見到她。」王子痛苦極了,絕望地從塔上
跳了下去。他掉進了刺叢裡,雖然沒有喪生,雙眼卻被刺扎瞎了。他漫無目的地
在森林裡走著,吃的只是草根和漿果,每天都為失去愛人而傷心地痛哭。他就這樣
痛苦地在森林裡轉了好幾年,最後終於來到了萵苣姑娘受苦的荒野。萵苣姑娘已
經生下了一對雙胞胎,一個兒子,一個女兒。王子聽到有說話的聲音,而且覺得
那聲音很耳熟,便朝那裡走去。當他走近時,萵苣姑娘立刻認出了他,摟著他的
脖子哭了起來。她的兩滴淚水潤溼了他的眼睛,使它們重新恢復了光明。他又能
像從前一樣看東西了。他帶著妻子兒女回到自己的王國,受到了人們熱烈的歡迎。
他們幸福美滿地生活著,直到永遠。

%----------------------------------------------------------------------------------------

\end{document}
