% for OPmac + plain luatexja.
\input luatexja-core.sty
\jfont\ming={I.MingCP:jfm=zh_TW/quanjiao} at 11pt
\jfont\mings={I.MingCP:jfm=zh_TW/quanjiao} at 9pt
\jfont\mingl={I.MingCP:jfm=zh_TW/quanjiao} at 12.2pt
\jfont\mingL={I.MingCP:jfm=zh_TW/quanjiao} at 14.2pt
\jfont\hei={Taipei Sans TC Beta Light:jfm=zh_TW/quanjiao} at 11pt
\jfont\heil={Taipei Sans TC Beta Light:jfm=zh_TW/quanjiao} at 12.2pt
\jfont\heiL={Taipei Sans TC Beta Light:jfm=zh_TW/quanjiao} at 14.2pt
\jfont\heib={Noto Sans CJK TC Medium:jfm=zh_TW/quanjiao} at 11pt
\jfont\heibl={Noto Sans CJK TC Medium:jfm=zh_TW/quanjiao} at 12.2pt
\jfont\heibL={Noto Sans CJK TC Medium:jfm=zh_TW/quanjiao} at 14.2pt
\jfont\heibh={Noto Sans CJK TC Bold:jfm=zh_TW/quanjiao} at 18.6pt
\jfont\kai={TW-MOE-Std-Kai:jfm=zh_TW/quanjiao} at 11pt
\jfont\kail={TW-MOE-Std-Kai:jfm=zh_TW/quanjiao} at 12.2pt
\jfont\kaiL={TW-MOE-Std-Kai:jfm=zh_TW/quanjiao} at 14.2pt
\jfont\kaih={TW-MOE-Std-Kai:jfm=zh_TW/quanjiao} at 18.2pt

% 配合 OPmac 可使用 \chap, \sec, \secc……等等方便指令。
% 暫時中文字的大小要另行指令。將來擬擴充 OPmac 和 luatexja 配合。
\input opmac
\margins/1 a4 (1,1,1,1)in  % A4 portrait
\fontfam[Linux Libertine]
\typosize[11/16.4]

\hyperlinks{\Blue}{\Green}
\insertoutline{CONTENTS} \outlines{0}

% 暫時不蓋掉 luatexja 的預設字型,直接指定使用。
\ming

\tit{\heibh OPmac和Lua\TeX{}ja的配合}\fnotemark1

\fnotetext{\mings 目前使用很笨拙的方式配合,期待能修正至「正常」一點。}

\nonum\notoc\sec{Table of Contents\heibL 目錄}
\maketoc

\nonum \sec{Introduction\heibL 簡介}

「老兵不死,只是逐漸凋零」。用plain \TeX{}的最大好處是,二十年前的文件,
現在仍然可以正常編譯,幾乎不必修改。沒錯!\TeX{}就是這麼的「頑固」。
這是\LaTeX{}文件所不能及的。\TeX{}是老兵,但卻是「老驥伏櫪,志在千里」,
絕非「逐漸凋零」。

\chap{\heibL 這裡是第一章}

這裡是第一章內文。

\sec{\heibL 這裡是第一節}

這裡是第一節內文。

\secc{\heibl 這裡是第一節第一小節}

這裡是第一節第一小節內文。本文的OPmac的設定:

\typosize[11/12.5]
\begtt
\input opmac
\margins/1 a4 (1,1,1,1)in  % A4 portrait
\fontfam[Linux Libertine]
\typosize[11/16.4]
\endtt

底下是數學式:
$$\eqalignno{
    a^2+b^2 &= c^2 \cr
          c &= \sqrt{a^2+b^2} & \eqmark \cr}$$

\typosize[11/16.4]
\nonum \sec{\heibL 這裡是沒有前置號碼的第二節}

這裡是沒前置號碼的第二節內文。

\bye

