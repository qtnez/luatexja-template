%%%%%%%%%%%%%%%%%%%%%%%%%%%%%%%%%%%%%%%%%
% Professional Newsletter Template
% LaTeX Template
% Version 1.0 (09/03/14)
%
% Created by:
% Bob Kerstetter (https://www.tug.org/texshowcase/) and extensively modified by:
% Vel (vel@latextemplates.com)
%
% 2021.05.03
% zh_TW(LuaLaTeX) version by Edward G.J. Lee <edt1023#gmail.com>
%
% This template has been downloaded from:
% http://www.LaTeXTemplates.com
%
% License:
% CC BY-NC-SA 3.0 (http://creativecommons.org/licenses/by-nc-sa/3.0/)
%
%%%%%%%%%%%%%%%%%%%%%%%%%%%%%%%%%%%%%%%%%

\documentclass[10pt]{article} % The default font size is 10pt; 11pt and 12pt are alternatives

\input{structure.tex} % Include the document which specifies all packages and structural customizations for this template

\begin{document}

%----------------------------------------------------------------------------------------
%	HEADER IMAGE
%----------------------------------------------------------------------------------------

\begin{figure}[H]
\centering\includegraphics[width=0.3\linewidth]{logo.png}
\end{figure}

%----------------------------------------------------------------------------------------
%	第一頁左邊框
%----------------------------------------------------------------------------------------

\begin{minipage}[t]{.30\linewidth} % Mini page taking up 30% of the actual page
\begin{mdframed}[style=sidebar,frametitle={}] % Sidebar box

%-----------------------------------------------------------

\hypertarget{contents}{\textbf{{\large 重點議題……}}} % \hypertarget provides a label to reference using \hyperlink{label}{link text}
\begin{itemize}
\item \hyperlink{firstnews}{第一則新聞標題} % These link to their appropriate sections in the newsletter
\item \hyperlink{secondnews}{第二則新聞標題}
\item \hyperlink{thirdnews}{第三則新聞標題}
\item \hyperlink{descriptivebox}{短語方框}
\item \hyperlink{quotation}{自定的引言}
\end{itemize}

\centerline {\rule{.75\linewidth}{.25pt}} % Horizontal line

%-----------------------------------------------------------

\textbf{左方框第一標題}

又東三百八十里,曰猨翼之山,其中多怪獸,水多怪魚,多白玉,多腹虫,多怪
蛇,多怪木,不可以上。

\begin{enumerate}
\item 又東三百里,曰堂庭之山,多棪木,多白猿,多水玉,多黃金。
\item 又東三百八十里,曰猨翼之山,其中多怪獸,水多怪魚,多白玉,多腹虫,多怪蛇,多怪木,不可以上。
\item 又東三百七十里,曰瞿父之山,無草木,多金玉。
\item 又東四百里,曰句餘之山,無草木,多金玉。
\end{enumerate}

%-----------------------------------------------------------

\textbf{左方框第二標題}

又東三百七十里,曰杻陽之山,其陽多赤金,其陰多白金。有獸焉,其狀如馬而
白首,其文如虎而赤尾,其音如謠,其名曰鹿蜀,佩之宜子孫。怪水出焉,而東流
注於憲翼之水。其中多玄龜,其狀如龜而鳥首虺尾,其名曰\href{http://www.example.com}{旋龜},其音如判木,佩
之不聾,可以為底。

\textbf{左方框第三標題}

又東三百里,曰青丘之山,其陽多玉,其陰多青䨼。有獸焉,其狀如狐而九尾,
其音如嬰兒,能食人,食者不蠱。有鳥焉,其狀如鳩,其音若呵,名曰灌灌,佩之不
惑。英水出焉,南流注於即翼之澤。其中多\href{http://www.example.com}{赤鱬},其狀如魚而人面,其音如鴛鴦,
食之不疥。

%-----------------------------------------------------------
\begin{center}
\captionof*{table}{表格標題名}
\begin{tabular}{llr}
\toprule
\multicolumn{2}{c}{姓名} \\
\cmidrule(r){1-2}
姓 & 名 & 級數 \\
\midrule
周 & 伯通 & $7.5$ \\
李 & 莫愁 & $2$ \\
\bottomrule
\end{tabular}
\end{center}
%-----------------------------------------------------------

\end{mdframed}
\end{minipage}\hfill % End the sidebar mini page
%
%----------------------------------------------------------------------------------------
%	第一頁主版面
%----------------------------------------------------------------------------------------
%
\begin{minipage}[t]{.66\linewidth} % Mini page taking up 66% of the actual page

\hypertarget{firstnews}{\heading{第一則新聞標題}{6pt}} % \hypertarget provides a label to reference using \hyperlink{label}{link text}

又東五百里,曰會稽之山,四方,其上多金玉,其下多砆石。勺水出焉,而南流注於湨。

\begin{center}
\parbox[t]{.70\linewidth}{\textit{又東三百四十里曰堯光之山,其陽多玉,其陰多金。有獸焉,其狀如人而彘鬣,穴居而冬蟄,其名曰猾褢,其音如斲木,見則縣有大繇。}}
\end{center}

\textit{《南次二經》}之首,曰柜山,西臨流黃,北望諸毗,東望長右。英水出焉,西南
流注於赤水,其中多白玉,多丹粟。有獸焉,其狀如豚,有距,其音如狗吠,其名
曰狸力,見則其縣多土功。有鳥焉,其狀如鴟而人手,其音如痺,其名曰鴸,其鳴
自號也,見則其縣多放士。

\begin{wrapfigure}[7]{l}[0pt]{0pt} % In-line figure with text wrapping around it
\includegraphics[width=0.3\textwidth]{placeholder.jpg}
\end{wrapfigure}

東五百里,曰\textit{禱過之山},其上多金玉,其下多犀、兕,多象。有鳥焉,其狀如鵁,
而白首、三足、人面,其名曰瞿如,其鳴自號也。泿水出焉,而南流注於海。其中
有虎蛟,其狀魚身而蛇尾,其音如鴛鴦,食者不腫,可以已痔。

又東五百里,曰丹穴之山,其上多金玉。丹水出焉,而南流注於渤海。有鳥焉,
其狀如雞,五采而文,名曰鳳皇,首文曰德,翼文曰義,背文曰禮,膺文曰仁,腹
文曰信。是鳥也,飲食自然,自歌自舞,見則天下安寧。

項籍少時,學書不成,去學劍,又不成。項梁怒之。籍曰:「書足以記名姓而已。
劍一人敵,不足學,學萬人敵。」於是項梁乃教籍兵法,籍大喜,略知其意,又不
肯竟學。項梁嘗有櫟陽逮,乃請蘄獄掾曹咎書抵櫟陽獄掾司馬欣,以故事得已。項
梁殺人,與籍避仇於\href{http://www.example.com/}{吳中}。吳中賢士大夫皆出項梁下。每吳中有大繇役及喪,項梁
常為主辦,陰以兵法部勒賓客及子弟,以是知其能。秦始皇帝游會稽,渡浙江,梁
與籍俱觀。籍曰:「彼可取而代也。」梁掩其口,曰:「毋妄言,族矣!」梁以此奇
籍。籍長八尺餘,力能扛鼎,才氣過人,雖吳中子弟皆已憚籍矣。

%-----------------------------------------------------------

\hypertarget{secondnews}{\heading{第二則新聞標題}{6pt}} % \hypertarget provides a label to reference using \hyperlink{label}{link text}

項籍者,下相人也,字羽。初起時,年二十四。其季父項梁,梁父即楚將項燕,
為秦將王翦所戮者也。項氏世世為楚將,封於項,故姓項氏。

\begin{itemize}
\item 又東五百里,曰咸陰之山,無草木,無水。
\item 又東三百五十里,曰羽山,其下多水,其上多雨,無草木,多蝮虫。
\item 又東五百里,曰僕勾之山,其上多金玉,其下多草木,無鳥獸,無水。
\end{itemize}

又東四百里,曰洵山,其陽多金,其陰多玉。有獸焉,其狀如羊而無口,不可殺
也,其名曰䍺。洵水出焉,而南流注於閼之澤,其中多芘蠃。

又東五百里,曰鹿吳之山,上無草木,多金石。澤更之水出焉,而南流注於滂
水。水有獸焉,名曰蠱雕,其狀如雕而有角,其音如嬰兒之音,是食人。

凡《南次二經》之首,自柜山至於漆吳之山,凡十七山,七千二百里。其神狀皆
龍身而鳥首。其祠:毛用一璧瘞,糈用稌。

《南次三經》之首,曰天虞之山,其下多水,不可以上。

又東四百里,曰\href{http://www.example.com/}{令丘之山},無草木,多火。其南有谷焉,曰中谷,條風自是出。
有鳥焉,其狀如梟,人面四目而有耳,其名曰顒,其鳴自號也,見則天下大旱。

\end{minipage} % End the main body - first page mini page

%----------------------------------------------------------------------------------------
%	第二頁主版面
%----------------------------------------------------------------------------------------

\begin{minipage}[t]{.66\linewidth} % Mini page taking up 66% of the actual page

\hypertarget{thirdnews}{\heading{第三新聞標題}{6pt}} % \hypertarget provides a label to reference using \hyperlink{label}{link text}

\begin{multicols}{2} % Two-column layout

\zhlipsum[6,7,8][name=xiangyu]

\BackToContents % Link back to the contents of the newsletter

\end{multicols}

%----------------------------------------------------------------------------------------
%	短語方框
%----------------------------------------------------------------------------------------

\begin{mdframed}[style=intextbox,frametitle={}] % Sidebar box

\hypertarget{descriptivebox}{\heading{短語方框}{0pt}} % \hypertarget provides a label to reference using \hyperlink{label}{link text}

又東三百七十里,曰侖者之山,其上多金玉,其下多青䨼。有木焉,其狀如穀而
赤理,其汗如漆,其味如飴,食者不飢,可以釋勞,其名曰白䓘,可以血玉。
\begin{enumerate}
\item 又東五百八十里,曰禺槀之山,多怪獸,多大蛇。
\item 右南經之山志,大小凡四十山,萬六千三百八十里。
\item 又東五百里,曰發爽之山,無草木,多水,多白猿。汎水出焉,而南流注於渤海。
\item 又東四百里,至於旄山之尾,其南有谷,曰育遺,多怪鳥,凱風自是出。
\item 又東四百里,至於非山之首,其上多金玉,無水,其下多蝮虫。
\item 又東五百里,曰陽夾之山,無草木,多水。
\item 又東五百里,曰灌湘之山,上多木,無草;多怪鳥,無獸。
\item 又東三百七十里,曰侖者之山,其上多金玉,其下多青䨼。
\end{enumerate}

\BackToContents % Link back to the contents of the newsletter

\end{mdframed}

%----------------------------------------------------------------------------------------
%	QUOTATION
%----------------------------------------------------------------------------------------

\hypertarget{quotation}{\large 自訂引言} % \hypertarget provides a label to reference using \hyperlink{label}{link text}

\begin{quote}
\textsl{『項籍者,下相人也,字羽。初起時,年二十四。其季父項梁,梁父即楚將項燕,
為秦將王翦所戮者也。項氏世世為楚將,封於項,故姓項氏。』} —— \textrm{班固}
\end{quote}

%----------------------------------------------------------------------------------------

\end{minipage}\hfill % End of the main body - second page mini page
\begin{minipage}[t]{.30\linewidth} % Mini page taking up 30% of the actual page

%----------------------------------------------------------------------------------------
%	第二頁右邊框
%----------------------------------------------------------------------------------------

\begin{mdframed}[style=sidebar,frametitle={}] % Sidebar box

\heading{規格}{0pt}

班固:

\begin{enumerate}
\item 楚左尹項伯者,項羽季父也,素善留侯張良。
\item 張良是時從沛公,項伯乃夜馳之沛公軍,私見張良,具告以事,欲呼張良與俱去。
\item 曰:「毋從俱死也。」張良曰:「臣為韓王送沛公,沛公今事有急,亡去不義,不可不語。」良乃入,具告沛公。
\item 沛公大驚,曰:「為之柰何?」張良曰:「誰為大王為此計者?」曰:「鯫生說我曰『距關,
毋內諸侯,秦地可盡王也』。故聽之。

\zhlipsum[10][name=trad]

\end{enumerate}

\BackToContents % Link back to the contents of the newsletter

\end{mdframed}\hfill

%----------------------------------------------------------------------------------------

\centering
\begin{minipage}[t]{.95\linewidth}
\textbf{聯繫資訊:}\\
雙手互搏資訊有限公司\\
黃藥師路,\\
桃花島\\
\href{http://www.example.com}{http://www.ornarejustoultricies.com}\\
\href{http://www.example.com}{http://www.elementumsapien.com}\\
\href{http://www.example.com}{http://www.nuncultrices.com}
\end{minipage}

%----------------------------------------------------------------------------------------

\end{minipage} % End of the sidebar mini page

%----------------------------------------------------------------------------------------

\end{document}
