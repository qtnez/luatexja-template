%%%%%%%%%%%%%%%%%%%%%%%%%%%%%%%%%%%%%%%%%
% Focus Beamer Presentation
% LaTeX Template
% Version 1.0 (8/8/18)
%
% This template has been downloaded from:
% http://www.LaTeXTemplates.com
%
% Original author:
% Pasquale Africa (https://github.com/elauksap/focus-beamertheme) with modifications by
% Vel (vel@LaTeXTemplates.com)
%
% 2021.04.30
% zh_TW(LuaLaTeX) version by Edward G.J. Lee <edt1023@gmail.com>
%
% Template license:
% GNU GPL v3.0 License
%
% Important note:
% The bibliography/references need to be compiled with bibtex.
%
%%%%%%%%%%%%%%%%%%%%%%%%%%%%%%%%%%%%%%%%%

%----------------------------------------------------------------------------------------
%  PACKAGES AND OTHER DOCUMENT CONFIGURATIONS
%----------------------------------------------------------------------------------------

\documentclass[hyperref={bookmarks=false}]{beamer}

\usetheme{focus} % Use the Focus theme supplied with the template
% Add option [numbering=none] to disable the footer progress bar
% Add option [numbering=fullbar] to show the footer progress bar as always full with a slide count

% Uncomment to enable the ice-blue theme
\definecolor{main}{RGB}{92, 138, 168}
\definecolor{background}{RGB}{240, 247, 255}

% for luatex-ja presentation and zh_TW fonts setting.
\usepackage[match]{luatexja-fontspec}
\setmainjfont[BoldFont=Noto Sans CJK TC Medium,
              YokoFeatures = {JFM = zh_TW/quanjiao},
              BoldItalicFont=I.Ngaan,
              ItalicFont=TW-MOE-Std-Kai]{I.MingCP}
\setsansjfont[BoldFont=Noto Sans CJK TC Medium,
              YokoFeatures = {JFM = zh_TW/quanjiao},
              BoldItalicFont=I.Ngaan,
              ItalicFont=TW-MOE-Std-Kai]{Taipei Sans TC Beta Light}
\setmonojfont[BoldFont=Noto Sans CJK TC Medium,
              YokoFeatures = {JFM = zh_TW/quanjiao},
              BoldItalicFont=I.Ngaan,
              ItalicFont=TW-MOE-Std-Kai]{cwTeXFangSong}
\setjfontfamily\iyan{I.Ngaan}
%\usepackage{libertinus} % libertinus font.

\newcommand{\zhtoday}{%
 \kansuji\year 年
 \kansuji\month 月
 \kansuji\day 日}

\renewcommand{\figurename}{圖}
\renewcommand{\tablename}{表}

%------------------------------------------------

\usepackage{booktabs} % Required for better table rules

%----------------------------------------------------------------------------------------
%   TITLE SLIDE
%----------------------------------------------------------------------------------------

\title{Focus: \\ 一個簡約主義的 Beamer 佈景主題}

\subtitle{次標題}

\author{周伯通 \\ 小龍女}

\titlegraphic{\includegraphics[scale=1.25]{Images/focuslogo.pdf}} % Optional title page image, comment this line to remove it

\institute{單位名稱 \\ 單位地址}

\date{\zhtoday}

%------------------------------------------------

\begin{document}

%------------------------------------------------

\begin{frame}
  \maketitle % Automatically created using the information in the commands above
\end{frame}

%----------------------------------------------------------------------------------------
%   SECTION 1
%----------------------------------------------------------------------------------------

\section{第一節} % Section title slide, unnumbered

%------------------------------------------------

\begin{frame}{\iyan 簡單的簡報}
  這是一個簡單的簡報。
\end{frame}

%------------------------------------------------

\begin{frame}[plain]{Plain Slide}
%  This is a slide with the plain style and it is numbered.
  這是易使用 plain style 的簡報(附有數字)。
\end{frame}

%------------------------------------------------

\begin{frame}[t]
%  This slide has an empty title and is aligned to top.
  這個頁面沒有標題,而且沿上排列。
\end{frame}

%------------------------------------------------

\begin{frame}[noframenumbering]{\iyan 沒有附上數字的頁面}
%  This slide is not numbered and is citing reference \cite{knuth74}.
  這是沒附上數字的頁面,並且有引用參文獻\cite{knuth74}。
\end{frame}

%------------------------------------------------

\begin{frame}{\iyan 排版及數式}
%  The packages \texttt{inputenc} and \texttt{FiraSans}\footnote{\url{https://fonts.google.com/specimen/Fira+Sans}}\textsuperscript{,}\footnote{\url{http://mozilla.github.io/Fira/}} are used to properly set the main fonts.
 \texttt{inputenc}及\texttt{FiraSans}\footnote{\url{https://fonts.google.com/specimen/Fira+Sans}}\textsuperscript{,}\footnote{\url{http://mozilla.github.io/Fira/}} 套件,用於設定適合的字型(和 \texttt{luatex-ja} 可能會衝突,小心使用)。
  \vfill
%  This theme provides styling commands to typeset \emph{emphasized}, \alert{alerted}, \textbf{bold}, \textcolor{example}{example text}, \dots
  這個佈景主題提供指令來設定來\emph{加強}、\alert{警訊}、\textbf{粗體}、\textcolor{example}{範例文字}……
  \vfill
  \texttt{FiraSans} 亦提供數理式子符號:
  \begin{equation*}
    e^{i\pi} + 1 = 0.
  \end{equation*}
\end{frame}

%----------------------------------------------------------------------------------------
%   SECTION 2
%----------------------------------------------------------------------------------------

\section{第二節}

%------------------------------------------------

\begin{frame}{\iyan 區塊}
%  These blocks are part of 1 slide, to be displayed consecutively.
  這些區塊都是屬於同一個 slide 頁面,是為了可以連續顯示。
  \begin{block}{一般區塊}
    這裡是一般區塊中的文字。
  \end{block}
  \pause % Automatically creates a new "page" split between the above and above + below
  \begin{alertblock}{警訊區塊}
    警訊 \alert{警訊文字}。
  \end{alertblock}
  \pause % Automatically creates a new "page" split between the above and above + below
  \begin{exampleblock}{實例文字區塊}
    實例 \textcolor{example}{實例文字內容}。
  \end{exampleblock}
\end{frame}

%------------------------------------------------

\begin{frame}{\iyan 多欄位}
  \begin{columns}
    \column{0.5\textwidth}
%      This text appears in the left column and wraps neatly with a margin between columns.
      這些文字會出現在左欄位,而且會在欄位邊界自動折行。

    \column{0.5\textwidth}
      \includegraphics[width=\linewidth]{Images/placeholder.jpg}
  \end{columns}
\end{frame}

%------------------------------------------------

\begin{frame}{\iyan 列舉}
  \begin{columns}[T, onlytextwidth] % T for top align, onlytextwidth to suppress the margin between columns
    \column{0.33\textwidth}
      項目列舉:
      \begin{itemize}
        \item 項目一
        \begin{itemize}
          \item 次項目 1.1
          \item 次項目 1.2
        \end{itemize}
        \item 項目二
        \item 項目三
      \end{itemize}

    \column{0.33\textwidth}
      數字列舉:
      \begin{enumerate}
        \item 數字一
        \item 數字二
        \begin{enumerate}
          \item 數字次項目
          \item 數字次項目
        \end{enumerate}
        \item 數字三
      \end{enumerate}

    \column{0.33\textwidth}
      敘述列舉:
      \begin{description}
        \item[第一項] 是。
        \item[第二項] 否。
      \end{description}
  \end{columns}
\end{frame}

%------------------------------------------------

\begin{frame}{\iyan 表格}
  \begin{table}
    \centering % Centre the table on the slide
    \begin{tabular}{l c}
      \toprule
      專業 & 平均薪資 \\
      \toprule
      \textbf{工程} & \textbf{\$66,521} \\
      電腦科學 & \$60,005\\
      數學 & \$61,867\\
      商業 & \$56,720\\
      人文及社會科學 & \$56,669\\
      農業及自然景觀 & \$53,565\\
      通訊 & \$51,448\\
      \midrule
      \textbf{所有專業平均} & \textbf{\$58,114}\\
      \bottomrule
    \end{tabular}
  \caption{這裡是你的表格標題}
  \end{table}
\end{frame}

%------------------------------------------------

\begin{frame}[focus]
  {\iyan 感謝使用 \textbf{Focus}!}
\end{frame}

%----------------------------------------------------------------------------------------
%   CLOSING/SUPPLEMENTARY SLIDES
%----------------------------------------------------------------------------------------

\appendix

\begin{frame}{\iyan 參考文獻}
  \nocite{*} % Display all references regardless of if they were cited
  \bibliography{example.bib}
  \bibliographystyle{plain}
\end{frame}

%------------------------------------------------

\begin{frame}{\iyan 備用頁面}
%  This is a backup slide, useful to include additional materials to answer questions from the audience.
  這是備用的頁面,應付聽眾額外的資料及答案。
  \vfill
% The package \texttt{appendixnumberbeamer} is used to refrain from numbering appendix slides.
   \texttt{appendixnumberbeamer} 可用於修正 Beamer 的附錄數字。
\end{frame}

%----------------------------------------------------------------------------------------

\end{document}
