% for OPmac + plain luatexja.
\input luatexja-core.sty
\jfont\ming={I.MingCP:jfm=zh_TW/quanjiao} at 11pt
\jfont\mingl={I.MingCP:jfm=zh_TW/quanjiao} at 12.2pt
\jfont\mingL={I.MingCP:jfm=zh_TW/quanjiao} at 14.2pt
\jfont\mings={I.MingCP:jfm=zh_TW/quanjiao} at 9pt
\jfont\hei={Taipei Sans TC Beta Light:jfm=zh_TW/quanjiao} at 11pt
\jfont\heil={Taipei Sans TC Beta Light:jfm=zh_TW/quanjiao} at 12.2pt
\jfont\heiL={Taipei Sans TC Beta Light:jfm=zh_TW/quanjiao} at 14.2pt
\jfont\heib={Noto Sans CJK TC Medium:jfm=zh_TW/quanjiao} at 11pt
\jfont\heibl={Noto Sans CJK TC Medium:jfm=zh_TW/quanjiao} at 12.2pt
\jfont\heibL={Noto Sans CJK TC Medium:jfm=zh_TW/quanjiao} at 14.2pt
\jfont\heibh={Noto Sans CJK TC Bold:jfm=zh_TW/quanjiao} at 18.6pt
\jfont\kai={TW-MOE-Std-Kai:jfm=zh_TW/quanjiao} at 11pt
\jfont\kail={TW-MOE-Std-Kai:jfm=zh_TW/quanjiao} at 12.2pt
\jfont\kaiL={TW-MOE-Std-Kai:jfm=zh_TW/quanjiao} at 14.2pt
\jfont\kaih={TW-MOE-Std-Kai:jfm=zh_TW/quanjiao} at 18.6pt


% 配合 OPmac 可使用 \chap, \sec, \secc……等等方便指令。
% 暫時中文字的大小要另行指令。將來擬擴充 OPmac 和 luatexja 配合。
\input opmac
\margins/1 a4 (1,1,1,1)in  % A4 portrait
\fontfam[Linux Libertine]
\typosize[11/18.4]

\ming
