% 給網路小說編譯成 pdf 檔使用,a4paper 要指定,不然常會被一些 package 更改
% 掉,例如 hyperref 會更改成 letters。
\documentclass[11pt,a4paper]{article}
\usepackage[myfont]{ltj-zhfonts}
% unicode-math 速度會慢。
%\usepackage{unicode-math}
% osf(old style font) 可以顯示上下參差不齊的數字,但數學模式仍然是使用正常,
% 這 style 會變動 baseline,因此不適合數學模式。
%\usepackage[osf]{libertinus}
\usepackage{libertinus}

\textheight=27.5cm
\headheight=0.2in
\footskip=0.2in
\addtolength{\topmargin}{-1.4in}
\renewcommand{\baselinestretch}{1.46}

% 中文縮排二個中文字。
\newlength{\zhind}
\settowidth{\zhind}{終極}
\setlength{\parindent}{\zhind}

% 將 section counter 關掉,這樣就不必使用  \section*{}
\setcounter{secnumdepth}{0}
\usepackage[pdfstartview={FitH},
            pdfencoding={auto},
            pdfauthor={元.施耐庵},
            pdfsubject={水滸傳},
            pdftitle={水滸傳},
            pdfkeywords={古典四大文學名著、六才子書之一},
            bookmarksopen={true}]{hyperref}

\title{\iyan 水滸傳}
\author{元.施耐庵。羅貫中整理\thanks{作者歷來有爭議,一般認為是施耐庵所著,而羅貫中則做了整理,金聖歎刪減為七十回本。}}
\date{元末明初(全書定型於明朝)}
\begin{document}

\maketitle

\subsection{楔子 張天師祈禳瘟疫 洪太尉誤走妖魔}

話說大宋仁宗天子在位,嘉佑三年三月三日五更三點,天子駕坐紫哀殿,受百官朝賀。但見:

祥雲迷鳳閣,瑞氣罩龍樓。含煙禦柳拂籃旗,帶露宮花迎劍戟。天香影里,玉吞珠履聚丹墀;仙樂聲中,繡襖錦衣扶御駕。珍珠簾卷,黃金殿上現金輿;鳳羽扇開,白王階前停寶輦。隱隱凈鞭三下響,層層文武兩班齊。

當有殿頭官喝道:「有事出班早奏,無事卷簾退朝。」只見班部叢中,宰相趙哲、參政文彥博出班奏曰:「目今京師瘟疫盛行,傷損軍民甚多。伏望陛下釋罪寬恩,省刑薄稅,祈禳天災,救濟萬民。」天子聽奏,急敕翰林院隨即草詔:一面降赦天下罪囚,應有民間稅賦悉皆赦免;一面命在京宮觀寺院,修設好事禳災。不料其年瘟疫轉盛。仁宗天子聞知,龍體下安,復會百官計議。向那班部中,有一大臣越班啟奏。天子看時,乃是參知政事範仲淹。拜罷起居,奏曰。「目今天災盛行,軍民塗炭,日夕不能聊生。以臣愚意,要禳此災,可宣嗣漢天師星夜臨朝,就京師禁院修設三千六百分羅天大醮,奏聞上帝,可以禳保民間瘟疫。」仁宗天子準奏。急令翰林學士草詔一道,天子御筆親書,井降御香一柱,欽差內外提點殿前大尉洪信為天使,前往江西信州龍虎山,宣請嗣漢夭師張真人星夜來朝,祈禳瘟疫。就金殿上焚起御香,親將丹詔忖與洪大尉,即便登程前去。

洪信領了聖敕,辭別天於,背了詔書,盛了御香,帶了數十人,上了鋪馬,一行部從,離了東京,取路徑投信州貴溪縣來。但見:

遙山疊翠,遠木澄清。奇花綻錦繡鋪林,嫩柳舞金絲拂地。風和日暖,時過野店山村;路直沙平,夜宿郵亭驛館。羅衣蕩漾紅塵內,駿馬驅馳紫陌中。

且說大尉洪信資擎御書,一行人從上了路途,不止一日,來到江西信州。大小官員出郭迎接,隨即差人報知龍虎山上清宮住持道眾,準備接詔。次日,眾位官同送太尉到於龍虎山下。只見上清宮許多道眾,鳴鐘擊鼓,香花燈燭,幢幡寶蓋,一派仙樂,都下山來迎接丹詔,直至上清宮前下馬。太尉看那官殿時,端的是好座上清宮。但見:

青松屈曲,翠柏陰森。門懸敕額金書,戶列靈符玉篆。虛皇壇畔,依稀垂柳名花;煉藥爐邊,掩映蒼松老檜。左壁廂天丁力士,參隨著大乙真君;右勢下玉女金童,簇捧定紫微大帝。披發仗劍,北方真武踏龜蛇;權履頂冠,南極老人伏龍虎。前排二十八宿星君,後列三十二帝天子。階砌下流水語謾,墻院後好山環繞。鶴生丹頂,龜長綠毛。樹梢頭獻果蒼猿,莎草內銜芝白鹿。三清殿上,嗚金鐘道士步虛;四聖堂前,敲玉磐真人禮斗,獻香臺砌,彩霞光射碧琉璃;召將瑤壇,赤日影搖紅瑪淄。早來門外祥雲現,疑是天師送老君。

當下上至住持真人,下及道童侍從,前迎後引,接至三清殿上,請將詔書居中供養著。洪大尉便間監宮真人道:「天師今在何處?」住持真人向前享道:「好教大尉得知:這代祖師號曰虛靖天師,性好清高,倦於迎送,自向尤虎山頂,結一茅庵,修真養性,因此下住本宮。」太尉道:「目今天子宣詔,如何得見?」真人答道:「吝享已詔敕權供在殿上,貧道等亦不敢開讀。且請大尉到方丈獻茶,再煩汁議。」當時將丹詔供養在三清毆上,與眾官都到方丈,太尉居中坐下,執事人等獻茶,就進齋供,水陸俱備。

齋罷,大尉再間真人道:「既然天師在山頂庵中,何下著人請將下來相見,開宣丹詔?」真人稟道:「這代祖師雖在山頂,其實道行非常,能駕霧興雲,蹤跡不定。貧道等如常亦難得見,怎生教人請得下來?」太尉道,」似此如何得見!國今京師瘟疫盛行,今上天子特遣下官,貴捧禦書丹詔,親奉尤香,來請天師,要做三千六百分羅天大酸,以被天災,救濟萬民。似此怎生奈何:」真人享道:「天子要救萬民,只徐是大尉辦一點志誠心,齋戒沐浴,更換布衣,休帶從人,自背詔書,焚燒禦香,步行上山禮拜,叩請天師,方許得見。如若心不志誠,空走一遭」亦難得見。」大尉聽說,便道:「俺從京師食素到此,如何心不志誠?既然恁地,依著你說,明日絕早上山。」當晚各自權歇。

次日五更時分,眾道士起來,備下香湯,請大尉起來沐浴,換了一身新鮮布衣,腳下壽上麻鞋草履,吃了素齋,取過丹詔,用黃羅包袱背在脊梁上,手裡提著銀手爐,降降地燒著御香。許多道眾人等,送到後山,指與路徑。真人又稟道:「太尉要救萬民,休生退悔之心,只顧志誠上去。」太尉別了眾人,口誦天尊寶號,縱步上山來。

將至半山,望見大頂直侵霄漢,果然好座大山。正是。

根盤地角,頂接天心。遠觀磨斷亂雲痕,近看平吞明月魄。高低不等謂之山,側石通道謂之蛐,孤嶺崎嶇謂之路,上面平極謂之頂,頭圓下壯謂之巒,藏虎藏豹謂之穴,隱風隱云謂之巖,高人隱居謂之洞,有境有界謂之府,樵人出沒謂之徑,能通車馬謂之道,流水有聲謂之洞,古渡源頭謂之溪,巖崖滴水謂之泉」左壁為掩,右壁為映。出的是雲,納的是霧「錐尖象小,崎峻似峭,懸空似險,削磁如平。千峰競秀,萬壑爭流。瀑布斜飛,藤蘿倒掛。虎嘯時風主穀口,猿啼時月墜山腰。恰似青黛雜成千塊玉,碧紗籠罩萬堆煙。

這洪太尉獨自一個,行了一回,盤坡轉徑,攬葛攀藤。

約莫走過了數個山頭,三二里多路,看看腳酸腿軟,正走不動,口裡不說,肚裏躊躇,心中想道:「我是朝廷貴官,在京師時重捆而臥,列鼎而食,尚兀自倦怠,何曾穿草鞋,走這般山路!知他天師在那裡,卻教下官受這般苦!」又行不到三五十步,掇著肩氣喘。

只見山凹裡起一陣風,風過處,向那松樹背後奔雷也似吼一聲,撲地跳出一個吊猜白額錦毛大蟲來。洪太尉吃了一驚,叫聲:「阿籲!」撲地望後便倒。偷眼看那大蟲時,但見:

毛披一帶黃金色,爪露銀鉤十八隻。睛如閃電尾如鞭,口似血盆牙似就。

伸腰展臂勢猙獰,擺尾搖頭聲霹靂。山中狐兔盡潛藏,澗下樟袍皆斂跡。

那大蟲望著洪太尉,左盤右旋,咆哮了一口,托地望後山坡下跳了去。洪大尉倒在樹根底下,唬的三十六個牙齒捉對兒廝打,那心頭一似十五個吊桶,七上八落的響,渾身卻如中風麻木,兩腿一似鬥敗公雞,口裡連聲叫苦。大蟲去了一盞茶時,方才鴨將起來,再收拾地上香爐,還把龍香燒著,再上山來,務要尋見天師。又行過三五十步,口裡唄了數口氣,怨道:「皇帝御限,差俺來這裏,教我受這場驚恐!」說猶未了,只覺得那裡又一陣風。吹得毒氣直沖將來。大尉定睛看時,山邊竹藤裡箴絞地響,搶出一條吊桶大小、雪花也似蛇來。大尉見了,又吃一驚,撇了手爐,叫一聲:「我今番死也!」望後便倒在盤舵石邊。微閃開眼看那蛇時,但見:

昂首驚諷起,掣目電光生。動蕩則拆峽倒岡,呼吸則吹雲吐霧。鱗甲亂分千片玉,尾梢斜卷一堆銀。那條大蛇徑搶到盤舵石邊,朝著洪大尉盤做一堆,兩隻眼迸出金光,張開巨口,吐出舌頭,噴那毒氣在洪大尉臉上。驚得太尉三魂蕩蕩,七魄悠悠。那蛇看了洪大尉一回,望山下一溜,卻早不見了。大尉方才爬得起來,說道:「慚愧!驚殺下官!」看身上時,寒粟子比滑燦兒大小。口裡駕那道士:「叵耐無禮,戲弄下官,教俺受這般驚恐!若山上尋下見天師,下去和他別有話說。」再拿了銀提爐,整頓身上詔敕並衣服中幀,卻待再要上山去。

正欲移步,只聽得松樹背後隱隱地笛聲吹響,漸漸近來。大尉定睛看時,但見那一個道童,倒騎著一頭黃牛,橫吹著一管鐵笛,轉出山凹來。大尉看那道童時,但見:頭縮兩枚丫舍,身穿一領青衣。腰間絳結草來編,腳下芒鞋麻間隔。明眸皓齒,飄飄並不染塵埃;綠鬢朱顏,耿耿全然無俗態。

昔日呂侗賓有首牧童詩道得好:

草鋪橫野六七里,笛弄晚風三四聲。

歸來飽飯黃昏後,不脫蓑衣臥月明。

只見那個道童,笑吟吟地騎著黃牛,橫吹著那管鐵笛,正過山來。洪大尉見了,便喚那個道童:」你從那裡來?認得我麼?」道童不睬,只顧吹笛。大尉連間數聲,道童呵呵大笑,拿著鐵笛,指著洪大尉說道:「你來此問,莫非要見天師麼什大尉大驚,便道:「你是牧童,如何得知?」道童笑道:「我早間在草庵中伏侍天師,聽得天師說道:「今上皇帝差個洪大尉責擎丹詔御香,到來山中,宣我往東京做三千六百分羅天大酷,祈攘天下瘟疫。我如今乘鶴駕雲去也。」這早晚想是去了,不在庵中。你休上去,山內毒蟲猛獸極多,恐傷害了你性命。」大尉再阿道:「你不要說謊?」道童笑了一聲,也不回應,又吹著鐵笛轉過山坡去了。太尉尋思道:「這小的如何盡知此事?想是天師分付他,已定是了。」欲侍再上山去,」方才驚唬的苦,爭些兒送了性命,不如下山去罷。」

大尉拿著提爐,再尋舊路,奔下山來。眾道士接著,請至方丈坐下,真人便間太尉道:」曾見夭師麼?」大尉說道:「我是朝廷中貴官,如何教俺走得山路,吃了這般辛苦,爭些兒送了性命!為頭上至半山裏,跳出一只吊睛白額大蟲,驚得下官魂魄都沒了。又行不過一個山嘴,竹藤裡搶出一條雪花大蛇來,盤做一堆,攔住去路。若不是俺福分大,如何得性命回京?盡是你這道眾,戲弄下官!」真人復道:」貧道等怎敢輕慢大臣?這是祖師試抨大尉之心。本山雖有蛇虎,並不傷人,」大尉又道:「我正走下動,方欲再上山坡,只見松樹傍邊轉出一個道童,騎著一頭黃牛,吹著管鐵笛,正過山來。我便間他:』那裡來?識得俺麼?,他道:『已都知了。』說天師分付,早晨乘鶴駕雲望東京去了,下官因此回來。」真人道:「大尉可惜錯過,這個牧童正是天師!」大尉道:「他既是天師,如何這等狠催?」真人答道:「這代天師非同小可,雖然年幼,其實道行非常。他是額外之人,四方顯化,極是靈驗。世人皆稱為道通祖師。」洪大尉道:「我直如此有眼不識真師,當面錯過!」真人道:「大尉且請放心,既然祖師法旨道是去了,比及大尉回京之日,這場酌事祖師已都完了。」大尉見說,方才放心。真人一面教安排筵宴,管待大尉;請將丹詔收藏於御書匣內,留在上清宮中,尤香就三清殿上燒了。當日方大內大排齋供,設宴飲酌。至晚席罷,止宿到曉。

次日早膳已後,真人道眾並提點執事人等請大尉游山。大尉大喜。許多人從跟隨著,步行出方丈,前面兩個道童引路,行至宮前宮後,看玩許多景致。三清殿上,富貴不可盡言。左廊下,九天殿、紫微殿、北極殿;右廊下,太乙殿、三官毆、驅邪殿,諸宮看遍。

行到右廊後一所去處,洪太尉看時,另外一所毆字:一遭都是搗椒紅泥墻,正面兩扇朱紅棍予,門上使著胳膊大鎖鈦著,交叉上面貼著十數道封皮,封皮上又是重重疊疊使著朱印。棺前一面朱紅漆金字牌額,上書四個金字,寫道:「伏魔之殿」。大尉指著門道:「此殿是甚麼去處?」真人答道:「此乃是前代老租天師,鎖鎮魔王之殿,」太尉又問道:「如何上面重重疊疊貼著許多封皮?」真人答道:「此是老祖大唐洞玄國師封鎖魔王在此。但是經傳一代天師,親手便添一道封皮,使其子子孫孫下敢妄開。走了魔君,非常利害。今經八九代祖師,誓丁敢開。鎖用銅汁漁鑄,誰知裡面的事,小道自來往持本宮三十餘年,也只聽聞。」

洪大尉聽了,心中驚怪,想道:「我且試看魔王一看。」便對真人說道:「你且開門來,我看魔王甚麼模樣。」真人告道:「大尉,此毆決下敢開!先祖天師叮嚀告戒:『今後潛入,不許擅開。,」大尉笑道:」胡說!你等要妄生怪事,煽惑百姓良民,故意安排這等去處,假稱鋇鎮魔王,顯耀你們道術。我讀一鑒之書,何曾見鎖魔之法?神鬼之道,處隔幽冥,我不信有魔王在內」快快與我打開,我看龐王如何。」真人三回五次稟說:「此殿開不得,恐惹利害,有傷於人。」大尉大怒,指著道眾說道:「你等不開與我看,回到朝廷,先奏你們眾道土阻當宣詔,違別聖旨,不令我見天師的罪犯;後奏你等私設此殿,假稱鎖鎮庇王,煽惑軍民百姓。把你都追了度胖,刺配遠惡軍州受苦。」真人等懼怕大尉權勢,只得喚幾個人工道人來,先把封皮揭了,將鐵錘打開大鎖。

眾人把門推開,看裏面對,黑洞洞地,但見:昏昏默默,杏奮冥冥。數百年不見太陽光,億萬載難瞻明月影。不分南北,怎辨東西。黑煙召霄撲人寒,冷氣陰陰侵體顫。人跡下到之處,妖精往來之鄉。閃開雙目有如盲,伸出兩手不見掌。常如三十夜,卻似五更時。

眾人一齊都到殿內,黑暗暗不見一物。太尉教從人取十數個人把點著,將來打一照時,四邊井無別物,只中央一個石碑,約高五六尺,下面石龜跌坐,大半陷在泥里。照那碑閹上時,前面都是龍章鳳篆,天書符篆,人皆不識。照那碑後時,卻有四個真字大書,鑿著「遇洪而開」。卻不是一來夭罡星合當出世,二來宋朝必顯忠良,三來湊巧遇著洪信。豈不是無數!洪大尉看了這四個字,大喜,便對真人說道:「你等阻當我,卻怎地數百年前已注我姓字在此?『遇洪而開』,分明是教我開看,卻何妨!我想這個日王,都。只在石碑底下。汝等從人與我多喚幾個人工人等,將鋤頭鐵鍬來掘開。」真人慌忙諫道:」大尉,不可掘動!恐有利害,傷犯千人,下當穩便。」大尉大怒,喝道:「你等道眾,省得甚麼!卿L分明鑿著遇我教開,你如何阻當?快與我喚人來開。」真人又三回五次稟道:「恐有下好。」大尉那裡肯聽?只得聚集眾人,先把石碑放倒,一齊並力掘那石龜,半日方才掘得起。又掘下去,約有三四尺深,見一片大青石板,可方丈圍。洪大尉叫再掘起來。真人又苦享道:「不可掘動!」大尉那裡肯聽?眾人只得把石板一齊打起,看時,百板底下卻是一個萬丈深淺地穴。只見穴內刮刺刺一聲響亮,那響非同小可,恰似:

天摧地塌,岳撼山崩。錢塘江上,潮頭浪擁出海門來;泰華山頭,巨靈神一劈山峰碎。共工奮怒,去盔撞倒了不周山;力士施咸,飛錘擊碎了始皇輦。一風憎折於竿竹,十萬軍中半夜雷。

那一聲響亮過處,只見一道黑氣,從穴裏滾將起來,掀塌了半個殿角。那道黑氣直沖上半天裏,空中散作百十道金光,望四面八方去了。眾人吃了一驚,發聲喊,都走了,撇下鋤頭鐵鍬,盡從殿內奔將出來,推倒擷翻無數。驚得洪大尉目睜口呆,罔知所措,面色如上。奔到廊下,只見真人向前叫苦不迭。太尉間道:「走了的卻是甚麼妖魔?」那真人言不過數句,話不過一席,說出這個緣由。有分教:一朝皇帝,夜眠下穩,晝食忘餐。直使宛予城中藏猛虎,蘿兒窪內聚神蚊。

畢竟尤虎山真人說出甚言語來?且聽下回分解。

\subsection{第一回 王教頭私走延安府 九紋龍大鬧史家村}

話說故宋,哲宗皇帝在時,其時去仁宗天子已遠,東京,開封府,汴梁,宣武軍便有一個浮浪破落戶子弟,姓高,排行第二,自小不成家業,只好刺槍使棒,最是得好腳氣球。

京師人口順,不叫高二,卻都叫他做高球。

綁來發跡,便將氣球那字去了「毛傍」,添作「立人」,改作姓高,名俅。

這人吹彈歌舞,刺槍使棒,相撲頑耍,亦胡亂學詩書詞賦;若論仁義禮智,信行忠良,卻是不會,只在東京城裏城外幫閑。

因幫了一個生鐵王員外兒子使錢,每日三瓦兩舍,風花雪月,被他父親在開封府裡告了一紙文狀,府把高俅斷了二十脊杖,送配出界發放,東京城裏人民不許容他在家宿食。

高俅無計奈何,只得來淮西,臨淮州,投奔一個開賭坊的閑柳大郎,名喚柳世權。

他平生專好惜客養閑人,招納四方幹隔澇子。

高俅投托得柳大郎家,一住三年。

綁來哲宗天子因拜南郊,感得風調雨順,放寬恩,大赦天下,那高俅在臨淮州因得了赦宥罪犯,思量要回東京。

這柳世權卻和東京城裏金梁橋下開生藥鋪的董將仕是親戚,寫了一封書札,收拾些人事盤纏,齎發高俅回東京投奔董將仕家過活。

當時高俅辭了柳大郎,背上包裹,離了臨淮州,迤邐回到東京,逕來金梁橋下董生藥家下了這一封書。

董將仕一見高俅,看了柳世權來書,自肚裡尋思道:「這高俅,我家如何安得著遮著他?若是個志誠老實的人,可以容他在家出入,也教孩兒們學些好;他卻是個幫閑破落戶,沒信的人,亦且當初有過犯來,被斷配的人,舊性必一肯改,若留住在家中,倒惹得孩兒們不學好了。」

待不收留他,又撇不過柳大郎面皮,當時只得權且歡天喜地相留在家宿歇,每日酒食管待。

住了十數日,董將仕思量出一個路數,將出一套衣服,寫了一封書簡,對高俅說道:「小人家下螢火之光,照人不亮,恐後誤了足下。我轉薦足下與小蘇學士處,久後也得個出身。足下意內如何?」

高俅大喜,謝了董將仕。

董將仕使個人將著書簡,引領高俅逕到學士府內。

門吏轉報。

小蘇學士出來見了高俅,看了來書。

知道高俅原是幫閑浮浪的人,心下想道:「我這裏如何安著得他?不如做個人情,他去駙王晉卿府裡做個親隨;人都喚他做小王都太尉,他便歡喜這樣的人。」

當時回了董將仕書札,留高俅在府裡住了一夜。

次日,寫了一封書呈,使個乾人送高俅去那小王都太尉處。

這太尉乃是哲宗皇帝妹夫,神宗皇帝的駙馬。

他喜愛風流人物,正用這樣的人;一見小蘇學士差人持書送這高俅來,拜見了便喜;收留高俅在府內做個親隨。

自此,高俅遭際在王都尉府中,出入如同家人一般。

自古道:「日遠日疏,日親日近。」

蚌一日,小王都太尉慶生辰,分付府中安排筵宴;專請小舅端王。

這端王乃是神宗天子第十一子,哲宗皇帝御弟,現掌東駕,排號九大王,是個聰明俊俏人物。

這浮浪子弟門風幫閑之事,無一般不曉,無一般不會,更無一般不愛;即如琴棋書畫,無所不通,踢球打彈,品竹調絲,吹彈歌舞,自不必說。

當日,王都尉府中準備筵宴,水陸俱備。

請端王居中坐定,太尉對席相陪。

酒進數杯,食供兩套,那端王起身凈手,偶來書院里少歇,猛見書案上一對兒羊脂玉碾成的鎮紙獅子,極是做得好,細巧玲瓏。

端王拿起獅子,不落手看了一回,道:「好!」

王都尉見端王心愛,便說道:「再有一個玉龍筆架,也是這個匠人一手做的,卻不在手頭,明日取來,一並相送。」

端王大喜道:「深謝厚意;想那筆架必是更妙。」王都尉道:「明日取出來送至宮中便見。」

端王又謝了。

兩個依舊入席。

飲宴至暮,盡醉方散。

端王相別回宮去了。

次日,小王都太尉取出玉龍筆架和兩個鎮紙玉獅子,著一個小靶子盛了,用黃羅包袱包了,寫了一封書呈,卻使高俅送去。

高俅領了王都尉鈞旨,將著兩般玉玩器,懷中揣著書呈,逕投端王宮中來。

把門官吏轉報與院公。

沒多時,院公出來問道:「你是那個府裡來的人?」

高俅施禮罷,答道:「小人是王駙馬府中特送玉玩器來進大王。」

院公道:「殿下在庭心裡和小逼門踢氣球,你自過去。」

高俅道:「相煩引進。」

院公引到庭門。

高俅看時,見端王頭戴軟紗唐巾;身穿紫繡龍袍;腰系文武雙穗條;把繡龍袍前襟拽起扎揣在條兒邊;足穿一雙嵌金線飛鳳靴;三五個小逼門相伴著蹴氣球。

高俅不敢過去沖撞,立在從人背後伺侯。

也是高俅合當發跡,時運到來;那個氣球騰地起來,端王接個不著,向人叢裏直滾到高俅身邊。

那高俅見氣球來,也是一時的膽量,使個「鴛鴦拐,」踢還端王。

端王見了大喜,便問道:「你是甚人?」

高俅向前跪下道:「小的是王都尉親隨;受東人使令,送兩般玉玩器來進獻大王。有書呈在此拜上。」

端王聽罷,笑道:「姐夫真如此掛心?」

高俅取出書呈進上。

端王開盒子看了玩器。

都遞與堂候官收了去。

那端王且不理玉玩器下落,卻先問高俅道:「你原來會踢氣球?你喚做甚麼?」高俅叉手跪覆道:「小的叫高俅,胡亂踢得幾腳。」

端王道:「好,你便下場來踢一回耍。」

高俅拜道:「小的是何等樣人,敢與恩王下腳!」

端王道:「這是齊雲社,名為天下圓,但何傷。」

高俅再拜道:「怎敢。」

三回五次告辭,端王定要他,高俅只得叩頭謝罪,解膝下場。

才幾腳,端王喝採,高俅只得把平生本事都使出來奉承端王,那身分,模樣,這氣球一似鰾膠黏在身上的!端王大喜,那肯放高俅回府去,就留在宮中過了一夜;次日,排個筵會,專請王都尉宮中赴宴。

卻說王都尉當日晚不見高俅回來,正疑思間,只見次日門子報道:「九大王差人來傳令旨,請太尉到宮中赴宴。」

王都尉出來見了乾人,看了令旨,隨即上馬,來到九大王府前,下了馬,入宮來見了端王。

端王大喜,稱謝兩般玉玩器,入席,飲宴間,端王說道:「這高俅踢得兩腳好氣球,孤欲索此人做親隨,如何?」

王都尉答道:「既殿下欲用此人,就留在宮中伏侍殿下。」

端王歡喜,執杯相謝。

二人又閑話一回,至晚席散,王都尉自回駙馬府去,不在話下。

且說端王自從索得高俅做伴之後,留在宮中宿食。

高俅自此遭際端王每日跟隨,寸步不離。

未兩個月,哲宗皇帝晏駕,沒有太子,文武百官商議,冊立端王為天子,立帝號曰徽宗,便是玉清教主微妙道君皇帝。

登基之後,一向無事,忽一日,與高俅道:「朕欲要抬舉你,但要有邊功方可升遷,先教樞密院與你入名。」

只是做隨駕遷轉的人。

綁來沒半年之間,直抬舉高俅做到殿帥府太尉職事。

高俅得做太尉,揀選吉日良辰去殿帥府里到任。

所有一應合屬公吏,衙將,都軍,監軍,馬步人等,盡來參拜,各呈手本,開報花名。

高殿帥一一點過,於內只欠一名八十萬禁軍教頭王進,--半月之前,已有病狀在官,患病未痊。

--不曾入衙門管事。

高殿帥大怒,喝道:「胡說!既有手本呈來,卻不是那廝抗拒官府,搪塞下官?此人即是推病在家!快與我拿來!」

隨即差人到王進家來捉拿王進。

且說這王進卻無妻子,只有一個老母,年已六旬之上。

牌頭與教頭王進說道:「如今高殿帥新來上任,點你不著,軍正司稟說染病在家,見有患病狀在官,高殿帥焦躁,那裡肯信,定要拿你,只道是教頭詐病在家。教頭只得去走一遭;若還不去,定連累小人了。」

王進聽罷,只得捱著病來;進殿帥府前,參見太尉,拜了四拜,躬身唱個喏,起來立在一邊。

高俅道:「你那廝便是都軍教頭王升的兒子?」

王進稟道:「小人便是。」

高俅喝道:「這廝!你爺是街上使花棒賣藥的!你省得甚麼武藝?前官沒眼,參你做個教頭,如何敢小覷我,不伏俺點視!你托誰的勢要推病在家安閑快樂?」王進告道:「小人怎敢;其實患病未痊。」

高太尉罵道:「賊配軍!你既害病,如何來得?」

王進又告道:「太尉呼喚,不敢不來。」

高殿帥大怒,喝令:「左右!拿下!加力與我打這廝!」

眾多牙將都是和王進好的,只得與軍正司同告道:「今日是太尉上任好日頭,權免此人這一次。」

高太尉喝道:「你這賊配軍!且看眾將之面饒恕你今日!明日卻和你理會!」王進謝罪罷,起來抬頭看了,認得是高俅;出得衙門,嘆口氣道:「我的性命今番難保了!俺道是甚麼高殿帥,卻原來正是東京幫閑的圓社高二!比先時曾學使棒,被我父親一棒打翻,三四個月將息不起。有此之仇,他今日發跡,得做殿帥府太尉,正待要報仇。我不想正屬他管!自古道,「不怕官,只怕管。」俺如何與他爭得?怎生奈何是好?」回到家中,悶悶不已,對娘說知此事。

母子二人抱頭而哭。

娘道:「我兒,「三十六著,走為上著。」只恐沒處走!」

王進道:「母親說得是。兒子尋思,也是這般計較。只有延安府老種經略相公鎮守邊庭,他手下軍官多有曾到京師的,愛兒子使槍棒,何不逃去投奔他們?那裡是用人去處,足可安身立命。」

當下母子二人商議定了。

其母又道:「我兒,和你要私走,只恐門前兩個牌軍,是殿帥府撥來伏侍你的,若他得知,須走不脫。」

王進道:「不妨。母親放心,兒子自有道理措置他。」

當下日晚未昏。

王進先叫張牌入來,分付道:「你先吃了些晚飯,我使你一處去幹事。」

張牌道:「教頭使小人那裡去?」

王進道:「我因前日患病許下酸棗門外岳廟裏香願,明日早要去燒炷頭香。你可今晚先去分付廟祝,教他來日早些開廟門,等我來燒炷頭香,就要三牲獻劉李王。你就廟裡歇了等我。」

張牌答應,先吃了晚飯,叫了安置。望廟中去了。

當夜母子二人收拾了行李衣服,細軟銀兩,做一擔兒打挾了;又裝兩個料袋袱駝,拴在馬上的。

等到五更,天色未明,王進叫起李牌,分付道:「你與我將這些銀兩去岳廟裡和張牌買個三牲煮熟在那裡等候;我買些紙燭,隨後便來。」

李牌將銀子望廟中去了。

王進自去備了馬,牽出後槽,將料袋袱駝搭上,把索子拴縛牢了,牽在後門外,扶娘上了馬;家中粗重都棄了;鎖上前後門。

挑了擔兒,跟在馬後,趁五更天色未明,乘勢出了西華門,取路望延安府來。且說z繭P軍買了福物煮熟,在廟等到已牌,也不見來。

李牌心焦,走回到家中尋時,只見鎖了門,兩頭無路,尋了半日並無有人。

看看待晚,嶽廟裡張牌疑忌,一直奔回家來,又和李牌尋了一黃昏。

看看黑了,兩個見他當夜不歸,又不見了他老娘。

次日,兩個牌軍又去他親戚之家訪問,亦無尋處。

兩個恐怕連累,只得去殿帥府首告:「王教頭棄家在逃,母子不知去向。」

高太尉見告,大怒道:「賊配軍在逃,看那廝待走那裡去!」

隨即押下文書,行開諸州各府捉拿逃軍王進。

二人首告,免其罪責,不在話下。

且說王教頭母子二人自離了東京,免不了饑餐渴飲,夜住曉行。

在路一月有餘,忽一日,天色將晚,王進挑著擔兒跟在娘的馬後,口裡與母親說道:「天可憐見!慚愧了我母子兩個脫了這天羅地網之厄!此去延安府不遠了,高太尉便要差拿我也拿不著了!」

母子二人歡喜,在路上不覺錯過了宿頭,「走了這一晚,不遇著一處村坊,那裡去投宿是好?……」正沒理會處,只見遠遠地林子里閃出一道燈光來。

王進看了,道:「好了!遮莫去那裡陪個小心,借宿一宵,明日早行。」

當時轉入林子裡來看時,卻是一所大莊院,一周遭都是土墻,墻外卻有二三百株大柳樹。

當時王教頭來到莊前,敲門多時,只見一個莊客出來。

王進放下擔兒,與他施禮。

莊客道:「來俺莊上有甚事?」

王進答道:「實不相瞞,小人母子二人貪行了些路程,錯過了宿店,來到這裏,前不巴村,後不巴店,欲投貴莊借宿一宵。明日早行,依例拜納房金。萬望周全方便!」

莊客答道:「既是如此,且等一等,待我去問莊主太公。肯時但歇不妨。」

王進又道:「大哥方便。」

莊客入去多時,出來說道:「莊主太公教你兩個入來。」

王進請娘下了馬。

王進挑著擔兒,就牽了馬,隨莊客到裡面打麥場上,歇下擔兒,把馬拴在柳樹上。

母子二人,直到草堂上來見太公。

那太公年近六旬之上,須發皆白,頭戴遮塵暖帽,身穿直縫寬衫,腰系皂絲條,足穿熟皮靴。

王進見了便拜。

太公連忙道:「客人休拜。你們是行路的人,辛苦風霜,且坐一坐。」

王進子母二敘禮罷,都坐定。

太公問道:「你們是那裡來的?如何昏晚到此?」

王進答道:「小人姓張,原是京師人。因為消折了本錢,無可營用,要去延安府投奔親眷。不想今日路上貪行了程途,錯過了宿店,欲投貴莊借宿一宵。來日早行,房金依例拜納。」

太公道:「不妨。如今世上人那個頂著房屋走哩。你母子二位敢未打火?」

--叫莊客,--「安排飯來。」

沒多時,就廳上放開條桌子。

莊客托出一桶盤,四樣菜蔬,一盤牛肉,鋪放桌上,先燙酒來篩下。

太公道:「村落中無甚相待,休得見怪。」

王進起身謝道:「小人母子無故相擾,此恩難報。」

太公道:「休這般說,且請吃酒。」

一面勸了五七杯酒,搬出飯來,二人吃了,收拾碗碟,太公起身引王進母子到客房裡安歇。

王進告道:「小人母親騎的頭口,相煩寄養,草料望乞應付,一並拜酬。」

太公道:「這個不妨。我家也有頭口騾馬,教莊客牽出後槽,一發喂養。」

王進謝了,挑那擔兒到客房裡來。

莊客點上燈火,一面提湯來洗了腳。

太公自回裡面去了。

王進母子二人謝了莊客,掩上房門,收拾歇息。

次日,睡到天曉,不見起來。

莊主太公來到客房前過,聽得王進老母在房裡聲喚。

太公問道:「客官,天曉好起了?」

王進聽得,慌忙出房來見太公,施禮說道:「小人起多時了。夜來多多攪擾,甚是不當。」

太公問道:「誰人如此聲喚?」

王進道:「實不相瞞太公說,老母鞍馬勞倦,昨夜心痛病發。」

太公道:「即然如此,客人休要煩惱,教你老母且在老夫莊上住幾日。我有個醫心痛的方,叫莊客去縣裡撮藥來與你老母親吃。教他放心慢慢地將息。」

王進謝了。

卑休絮繁。

自此,王進母子二人在太公莊上。

服藥,住了五七日。

覺道母親病奔痊了,王進收拾要行。

當日因來後槽看馬,只見空地上一個後生脫著,刺著一身青龍,銀盤也似一個面皮,約有十八九歲,拿條棒在那裡使。

王進看了半晌,不覺失口道:「這棒也使得好了,只是有破綻,嬴不得真好漢。」

那後生聽了大怒,喝道:「你是甚麼人,敢來笑話我的本事!俺經了七八個有名的師父,我不信倒不如你!你敢和我叉一叉麼?」

說猶未了,太公到來喝那後生:「不得無禮!」

那後生道:「叵耐這廝笑話我的棒法!」

太公道:「客人莫不會使槍棒?」

王進道:「頗曉得些。敢問長上,這後生是宅上何人?」

太公道:「是老漢的兒子。」

王進道:「既然是宅內小官人,若愛學時,小人點撥他端正,如何?」

太公道:「恁地時十分好。」

便教那後生:「來拜師父。」

那後生那裡肯拜,心中越怒道:「阿爹,休聽這廝胡說!若吃他嬴得我這條棒時,我便拜他為師!」

王進道:「小官人若是不當真時,較量一棒耍子。」

那後生就空地當中把一條棒使得風車兒似轉,向王進道:「你來!你來!怕你不算好漢!」

王進只是笑,不肯動手。

太公道:「客官,既是肯教小頑時,使一棒,何妨?」

王進笑道:「恐沖撞了令郎時,須不好看。」

太公道:「這個不妨;若是打折了手腳,亦是他自作自受。」

王進道:「怒無禮。」

去槍架上拿了一條棒在手裡,來到空地上使個旗鼓。

那後生看了一看,拿條棒滾將入來,逕奔王進。

王進托地拖了棒便走。

那後生輪著棒又趕入來。

王進回身把棒望空地裡劈將下來。

那後生見棒劈來,用棒來隔。

王進卻不打下來,對棒一掣,卻望後生懷裡直搠將來,只一繳。

那後生的棒丟在一邊,撲地望後倒了。

王進連忙撇了棒,向前扶住,道:「休怪,休怪。」

那後生爬將起來,便去傍邊掇條凳子納王進坐,便拜道:「我枉自經了許多師家,原來不直半分!師父,沒奈何,只得請教!」

王進道:「我母子二人連日在此攪擾宅上,無恩可報,當以效力。」

太公大喜,教那後生穿了衣裳,一同來後堂坐下;叫莊客殺一個羊,安排了酒食果品之類,就請王進的母親一同赴席。

四個人坐定,一面把盞。

太公起身勸了一杯酒,說道:「師父如此高強,必是個教頭;小兒「有眼不識泰山。」」王進笑道:「好不廝欺,俏不廝瞞。小人不姓張,俺是東京八十萬禁軍教頭王進的便是。這槍棒終日摶弄。為因新任一個高太尉,原被先父打翻,今做殿帥府太尉,懷挾舊仇,要奈何王進,小人不合屬他所管,和他爭不得,只得母子二人逃上延安府去投托老種經略相公勾當。不想來到這裏,得遇長上父子二位如此看待;又蒙救了老母病奔,連日管顧,甚是不當。既然令郎肯學時,小人一力奉教。只是令郎學的都是花棒,只好看,上陣無用。小人從新點撥他。」

太公見說了,便道:「我兒,可知輸了?快來再拜師父。」

那後生又拜了王進。

太公道:「教頭在上,老漢祖居在這華陰縣界,前面便是少華山。這村便喚做史家村,村中總有三四百家都姓史。老漢的兒子從小不務農業,只愛刺槍使棒;母親說他不得,一氣死了。老漢只得隨他性子,不知使了多少錢財投師父教他;又請高手匠人與他剌了這身花繡,肩胸膛,總有九條龍。滿縣人口順,都叫他做九紋龍史進。教頭今日既到這裏,一發成全了他亦好。老漢自當重重酬謝。」王進大喜道:「太公放心;既然如此說時,小人一發教了令郎方去。」

自當日為始,吃了酒食,留住王教頭母子二人在莊上。

史進每日求王教頭點撥十八般武藝,一一從頭指教。

史太公自去華陰縣中承當里正,不在話下。

不覺荏苒光陰,早過半年之上。

史進十八般武藝,--矛,錘,弓,弩,銃,鞭,簡,劍,鏈,撾斧,鉞並戈,戟,牌,棒與槍,扒,……一一學得精熟。

多得王進盡心指教,點撥得件件都有奧妙。

王進見他學得精熟了,自思在此雖好,只是不了;一日,想起來,相辭要上延安府去。

史進那裡肯放,說道:「師父只在此間過了。小弟奉養你母子二人以終天年,多少是好。」

王進道:「賢弟,多蒙仔好心,在此十之好;只恐高太尉追捕到來,負累了你,不當穩便;以此兩難。我一心要去延安府投著在老種經略處勾當。那裡是鎮守邊庭,用人之際,足可安身立命。」

史進並太公苦留不住,只得安排一個席筵送行,托出一盤--兩個段子,一百兩花銀--謝師。

史進收拾了擔兒。備了馬,母子二人相辭史太公。

王進請娘乘了馬,望延安府路途進發。

史進叫莊客挑了擔兒,親送十里之程,心中難舍。

史進當時拜別了師父,灑淚分手,和莊客自回。

王教頭依舊自挑了擔兒,跟著馬,母子二人自取關西路上去了。

卑中不說王進去投軍役。

只說史進回到莊上,每日只是打熬氣力;亦且壯年,又沒老小,半夜三更起來演習武藝,白日裡只在莊射弓走馬。

不到半載之間,史進父親--太公--染病奔證,數日不起。

史進使人遠近請醫士看治,不能痊可。

嗚呼哀哉,太公歿了。

史進一面備棺槨盛殮,請僧修設好事,追齋理七,拔太公;又請道士建立齋醮,超度升天,整做了十數壇好事功果道場,選了吉日良時,出喪安葬,滿y中T四百史家莊戶都來送喪掛孝,埋殯在村西山上祖墳內了。

史進進家自此無人管業。

史進又不肯務農,只要尋人使家生,較量槍棒。

自史太公死後,又早過了三四個月日。

時當六月中旬,炎天正熱,那一日,史進無可消遣,提個交床坐在打麥場柳陰樹下乘涼。

對面松林透過風來,史進喝採道:「好涼風!」

正乘涼哩,只見一個人探頭探腦在那裡張望。

史進喝道:「作怪!誰在那裡張俺莊上?」

史進跳起身來,轉過樹背後,打一看時,認得是獵戶兔李吉。

史進喝道:「李吉,張我莊內做甚麼?莫不是來相腳頭!」

李吉向前聲諾道:「大郎,小人要尋莊上矮邱乙郎吃碗酒,因見大郎在此乘涼,不敢過來沖撞。」

史進道:「我且問你,往常時你只是擔些野味來我莊上賣,我又不曾虧了你,如何一向不將來賣與我?敢是欺負我沒錢?」

李吉答道:「小人怎敢;一向沒有野味,以此不敢來。」

史進道:「胡說!偌大一個少華山,恁地廣闊,不信沒有個獐兒,兔兒?」

李吉道:「大郎原來不知。如今山上添了一伙強人,扎下一個山寨,聚集著五七百個小嘍羅,有百十匹好馬。為頭那個大王喚作「神機軍師」朱武,第二個喚做「跳澗虎」陳達,第三個喚做「白花蛇」楊春,這三個為頭打家劫舍。華陰縣裡禁他不得,出三千貫賞錢,召人拿他。誰敢上去拿他?因此上,小人們不敢上山打捕野味,那討來賣!」

史進道:「我也聽得說有強人。不想那廝們如此大弄。必然要惱人。李吉,你今後有野味時尋些來。」

李苦唱個喏自去了。

史進歸到廳前,尋思「這廝們大弄,必要來薅惱村坊。既然如此……」便叫莊客揀兩頭肥水牛來殺了,莊內自有造下的好酒,先燒了一陌「順溜紙,」便叫莊客去請這當村裏三四百史家村戶都到家中草堂上序齒坐下,教莊客一面把盞勸酒。史進對眾人說道:「我聽得少華山上有三個強人,聚集著五七百小嘍羅打家劫舍。這廝們既然大弄,必然早晚要來俺村中羅噪。我今特請你眾人來商議。倘若那廝們來時,各家準備。我莊上打起梆子,你眾人可各執槍棒前來救應;你各家有事,亦是如此。遞相救護,共保村坊。如果強人自來,都是我來理會。」

眾人道:「我等村農只靠大郎做主,梆子響時,誰敢不來。」

當晚眾人謝酒,各自分散回家,準備器械。

自此,史進修整門戶墻垣,安排莊院,設立幾處梆子,拴束衣甲,整頻刀馬,防賊寇,不在話下。

且說少華山寨中三個頭領坐定商議,為頭的神機軍師朱武,那人原是定遠人氏,能使兩口雙刀,雖無十分本事。

卻精通陣法,廣有謀略;第二個好漢,姓陳,名達,原是鄴城人氏,使一條出白點鋼槍;第三個好漢,姓楊,名春,蒲州解良縣人氏,使一口大桿刀。

當日朱武卻與陳達,楊春說道:「如今我聽知華陰縣裏出三千賞錢,召人捉我們,誠恐來時要與他廝殺。只是山寨錢糧欠少,如何不去劫擄些來,以供山寨之用?聚積些糧食在寨裏,防備官軍來時,好和他打熬。」

跳澗虎陳達道:「說得是。如今便去華陰縣裏先問他借糧,看他如何。」

白花蛇楊春道:「不要華陰縣去;只去蒲城縣,萬無一失。」

陳達道:「蒲城縣人戶稀少,錢糧不多,不如只打華陰縣;里人民豐富,錢糧廣有。」

楊春道:「哥哥不知。若是打華陰縣時,須從史家村過。那個九紋龍史進是個大蟲,不可去撩撥他。他如何肯放我們過去?」

陳達道:「兄弟懦弱!一個村坊,過去不得,怎地敢抵敵官軍?」

楊春道:「哥哥,不可小了他!那人端的了得!」

朱武道:「我也曾聞他十分英雄,說這人真有本事。兄弟,休去罷。」

陳達叫將起來,說道:「你兩個閉了烏嘴!「長別人志氣,滅自己威風!」他只是一個人,須不三頭六臂?我不信!」喝叫小嘍羅:「快備我的馬來!如今便先去打史家莊,後取豹陰縣!」

朱武、楊春再三諫勸。

陳達那裡肯聽,隨即披掛上馬,點了一百四五十小嘍羅,鳴鑼擂鼓,下山望史家村去了。

且說史進正在莊前整制刀馬,只見莊客報知此事。

史聽得,就莊上敲起梆子來。

那莊前,莊後,莊東,莊西,三四百家莊戶,聽得梆子響,都拖槍曳棒,聚起三四百人,一齊都到史家莊上。

看了史進,頭戴一字巾,身披朱紅甲;上穿青錦襖,下著抹綠靴;腰系皮搭,前後鐵掩心;一張弓,一壺箭,手裡拿一把三尖兩刃四竅八環刀。

莊客牽過那匹火炭赤馬。

史進上了馬,綽了刀,前面擺著三四十壯健的莊客,後面列著八九十村蠢的鄉夫及史家莊戶,都跟在後頭,一齊吶喊,直到村北路口。

那少華山陳達引了人馬飛奔到山坡下,將小嘍羅擺開。

史進看時,見陳達頭戴乾紅凹面巾,身披裏金生鐵甲;上穿一領紅衲襖,腳穿一對吊墩靴;腰系七尺攢線搭;坐騎一匹高頭白馬;手中橫著丈八點鋼矛。

小嘍羅趁勢便吶喊。

二員將就馬上相見。

陳達在馬上看著史進,欠身施禮。

史進喝道:「汝等殺人放火,打家劫舍,犯著彌天大罪,都是該死的人!你也須有耳朵!懊大膽!直來太歲頭上動土!」

陳達在馬上答道:「俺山寨裏欠少些糧,欲往華陰縣借糧;經由貴莊,假一條路,並不敢動一根草。可放我們過去,回來自當拜謝。」

史進道:「胡說!俺家現當里正,正要拿你這伙賊;今日倒來經由我村中過卻不拿你,倒放你過去,本縣知道,須連累於我。」

陳達道:「「四海之內,皆兄弟也;」相煩借一條路。」

史進道:「甚麼閑話!我便肯時,有一個不肯!你問得他肯便去!」

陳達道:「好漢,叫我問誰?」

史進道:「你問得我手裡這口刀肯,便放你去!」

陳達大怒道:「趕人不要趕上!休得要逞精神!」

史進也怒,輪手中刀,驟坐下馬,來戰陳達。

陳達也拍馬挺槍來迎史進。

兩個交馬,鬥了多時,史進賣個破綻,讓陳達把槍望心窩裡搠來;史進卻把腰閃,陳達和槍擷入懷裡來;史進輕舒猿臂,款扭狼腰,只一挾,把陳達輕輕摘離了嵌花鞍,款款揪住了線搭,只一丟,丟落地,那匹戰馬撥風也似去了。

史進叫莊客把陳達綁了。

眾人把小嘍羅一趕都走了。

史進回到莊上,把陳達綁在庭心內柱上,等待一發拿了那賊首,一並解官請賞;且把酒來賞了眾人,教且權散。

眾人喝採:「不枉了史大郎如此豪傑!」

休說眾人歡喜飲酒。

卻說朱武、楊春,兩個正在寨裏猜疑,捉摸不定,且教小嘍羅再去探聽消息。只見回去的人牽著空馬,奔到山前,只叫道:「苦也!陳家哥哥不聽二位哥哥所說,送了性命!」

朱武問其緣故。

小嘍羅備說交鋒一節,「怎當史進英雄!」

朱武道:「我的言語不聽,果有此禍!」

楊春道:「我們盡數都去與他死並,如何?」

朱武道:「亦是不可;他尚自輸了,你如何並得他過?我有一條苦計,若救他不得,我和你都休。」

楊春問道:「如何苦計?」

朱武附耳低言說道:「只除恁地,……」楊春道:「好計!我和你便去!事不宜遲!」

再說史進正在莊上忿怒未消,只見莊客飛報道:「山寨裏朱武,楊春自來了。」

史進道:「這廝合休!我教他兩個一發解官!快牽過馬來!」

一面打起梆子。

眾人早都到來。

史進上了馬,正待出莊門,只見朱武、楊春,步行已到莊前,兩個雙雙跪下,擎著四行眼淚。

史進下馬來喝道:「你兩個跪下如何說?」

朱武哭道:「小人等三個累被官司逼迫,不得已上山落草。當初發願道:「不求同日生,只願同日死。」

雖不及關,張,劉備的義氣,其心則同。

今日小弟陳達不聽好言,誤犯虎威,已被英雄擒捉在貴莊,無計懇求,今來逕就死。

望英雄將我三人一發解官請賞,誓不皺眉。

我等就英雄手內請死,並無怨心!」

史進聽了,尋思道:「他們直恁義氣!我若拿他去解官請賞時,反教天下好漢們恥笑我不英雄。自古道:「大蟲不吃伏肉。」

」史進道:「你兩個且跟我進來。」

朱武、楊春,並無懼怯,隨了史進,直到後廳前跪下,又教史進綁縛。

史進三四五次叫起來。

他兩個那裡肯起來。

「惺惺惜惺惺,好漢識好漢。」

史進道:「你們既然如此義氣深重,我若送了你們,不是好漢。我放陳達還你,如何?」

朱武道:「休得連累了英雄,不當穩便,寧可把我們解官請賞。」

史進道:「如何使得。你肯吃我酒食麼?」

朱武道:「一死尚然不懼,何況酒肉乎!」

當時史進大喜,解放陳達,就後廳上座置酒設席管待三人。

朱武,楊春,陳達,拜謝大恩。

酒至數杯,少添春色。

酒罷,三人謝了史進,回山去了。

史進送出莊門,自回莊上。

卻說朱武等三人歸到寨中坐下,朱武道:「我們非這條苦計,怎得性命在此?雖然救了一人,卻也難得史大郎為義氣上放了我們。過幾日備些禮物送去,謝他救命之恩。」

卑休絮繁,過了十數日,朱武等三人收拾得三十兩蒜條金,使兩個小嘍羅送去史家莊上,當夜敲門。

莊客報知,史進火急披衣,來到莊前,問小嘍羅:「有甚話說?」

小嘍羅道:「三個頭領再三拜覆,特使進獻些薄禮,酬謝大郎不殺之恩。不要推卻,望乞笑留。」

取出金子遞與。

史進初時推卻,次後尋思道:「既然好意送來,受之為當。」

叫莊客置酒管待小校吃了半夜酒,把些零碎銀兩賞了小校回山。

又過半月餘,朱武等三人在寨中商議擄掠得好大珠子,又使小嘍羅連夜送來莊上。

史進受了,不在話下。

又過了半月,史進尋思道:「也難得這三個敬重我,我也備些禮物回奉他。」次日,叫莊客尋個裁縫,自去縣裡買了三疋紅綿,裁成三領錦襖子;又揀肥羊煮了三個,將大盒子盛了,委兩個莊客送去。

史進莊上有個為頭的莊客王四,此人頗能答應官府,口舌利便,滿莊人都叫他做「賽伯當」史進教他一個得力的莊客,挑了盒擔,直送到山下。

小嘍羅問了備細,引到山寨裏見了朱武等。

三個頭領大喜,受了錦襖子並肥羊酒禮,把十兩銀子賞了莊客,每人吃了十數碗酒,下山同歸莊內,見了史進,說道:「山上頭領多多上覆」。

史進自此常常與朱武等三人往來。

不時間,只是王四去山寨裏送物事,不只一日。

寨裡頭領也頻頻地使人送金銀來與史進。

荏苒光陰,時遇八月中秋到來。

史進要和三人說話,約至十五夜來莊上賞月飲酒,先使莊客王四帶一封請書直至少華山上請朱武,陳達,楊春,來莊上赴席。

王四馳書逕到山寨裏,見了三位頭領,下了來書。

朱武看了大喜。

三個應允,隨即寫封回書,賞了王四五兩銀子,吃了十來碗酒。

王四下得山來,正撞著時常送物事來的小嘍羅,一把抱住,那裡肯放,又拖去山路邊村酒店裡吃了十數碗酒。

王四相別了回莊,一面走著,被山風一吹,酒卻湧上來,踉踉蹌蹌,一步一顛;走不得十里之路,見座林子,奔到裡面,望著那綠茸茸莎草地上撲地倒了。

原來兔李吉正在那坡下張兔兒,認得是史家莊上王四,趕入林子裡來扶他,那裡扶得動,只見王四搭里出銀子來。

李吉尋思道:「這廝醉了,……那裡討得許多?……何不拿他些?」

也是天罡星合當聚會,自是生出機會來,李吉解那搭,望地下只一抖,那封回書和銀子都抖出來。

李吉拿起,頗識幾字;將書拆開看時,見面寫著少華山朱武,陳達,楊春;中間多有兼文武的言語,卻不識得,只認得三個字。

李吉道:「我做獵戶,幾時能彀發跡?算命道我今年有大財,卻在這裡!豹陰縣裡現出三千貫賞錢捕捉他三個賊人。叵耐史進那廝,前日我去他莊上尋矮邱乙郎,他道我來相腳頭屣盤,--你原來倒和賊人來往!」

銀子並書都拿去了,華陰縣裡來出首。

卻說莊客王四一覺直睡到二更方醒,覺得看見月光微微照在身上,吃了一驚,跳將起來,卻見四邊都是松樹;便去腰裡摸時,搭和書都不見了;四下裡尋時,只見空搭在莎草上。

王四只管叫苦,尋思道:「銀子不打緊,這封回書卻怎生得好?……正不知被甚人拿去了?……」眉頭一縱,計上心來,自道:「若回去莊上說脫了回書,大郎必然焦躁,定是趕我出來;不如只說不曾有回書,那裡查照?」

計較定了,飛也似取路歸來莊上,卻好五更天氣。

史進見王四回來,問道:「你緣何方才歸來?」

王四道:「托主人福蔭,寨中三個頭領都不肯放,留住王四吃了半夜乃,因此回來遲了。」

史進又問:「曾有回書麼?」

王四道:「三個頭領要寫回書,卻是小人道,「三位頭領既然準時赴席,何必回書?小人又有杯酒,路上恐有些失支脫節,不是耍處。」」史進聽了大喜,說道:「不枉了諸人叫你「賽伯當!」真個了得!」

王四應道:「小人怎敢差遲,路上不曾住腳,一直奔回莊上。」

史進道:「既然如此,教人去縣裡買些果品案酒伺候。」

不覺中秋節至。

是日晴明得好。

史進當日分付家中莊客宰了一腔大羊,殺了百十個雞鵝,準備下酒食筵宴。

看看天色晚來,少華山上朱武,陳達,楊春,三個頭領分付小嘍羅看守寨柵,只帶三五個做伴,將了樸刀,各跨口腰刀,不騎鞍馬,步行下山,逕來到史家莊上。

史進接著,各敘禮罷,請入後園。

莊內己安排下筵宴。

史進請三位頭領上坐,史進對席相陪,便叫莊客把前後莊門拴了,一面飲酒。莊內莊客輪流把盞,一邊割羊勸酒。

酒至數杯,卻早東邊推起那輪明月。

史進和三個頭領敘說舊話新言。

只聽得墻外一聲喊起,火把亂明。

史進大驚,跳起身來道:「三位賢友且坐,待我去看!」

叭叫莊客:「不要開門!」

掇條梯子上墻打一看時,只見是華陰縣尉在馬上,引著兩個都頭,帶著三四百士兵,圍住莊院。

史進及三個頭領只管叫苦。

外面火光中照見鋼叉,樸刀,五股寸,留客住,擺得似麻林一般。

兩個都頭口裡叫道:「不要走了強賊!」

不是這伙人來捉史並三個頭領,怎地教史進先殺了一二個人,結識了十數個好漢?直教,蘆花深處屯兵士,荷葉陰中治戰船。

畢竟史進與三個頭領怎地脫身,且聽下回分解。

\end{document}
